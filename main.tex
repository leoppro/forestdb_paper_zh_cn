%!TEX program = xelatex
\documentclass[lang=cn]{cls/elegantpaper}

\title{ForestDB: A Fast Key-Value Storage System for Variable-Length String Keys}
\author{Jung-Sang Ahn, Chiyoung Seo, Ravi Mayuram, Rahim Yaseen, Jin-Soo Kim, and Seungryoul Maeng}

\begin{document}

\maketitle

\begin{abstract}
\noindent \sffamily 基于持久存储的键值映射数据对NoSQL数据库而言是非常重要的组成部分。大多数键值存储引擎使用树型结构建立数据索引,但是他们的时间和空间开销随着键长度的增加而急剧增大。这同时也影响着对整体吞吐量至关重要的合并和压缩操作的开销。这篇论文介绍了一个单节点大规模NoSQL数据库存储引擎,我们将其命名为ForestDB。ForestDB使用HB$^+$字典树索引方案,这是一种基于磁盘的结合字典树与B$^+$树的新型混合索引结构。对比于其他树型结构,它索引性能较高。即便是键非常长,在空间内随机分布,HB$^+$字典树也可以以较低的磁盘访问量获取任意长度的键。我们的评测结果显示,在每秒操作数与每更新操作磁盘写入数两个性能指标上,ForestDB都显著的优于Couchbase Server、LevelDB和RocksDB现有的键值存储引擎。\\

\noindent \sffamily {\bfseries 关键词\quad} 键值存储系统\quad NoSQL\quad 索引结构\quad B$^+$树
\end{abstract}

\section{INTRODUCTION}

\newpage

\section{BACKGROUND}
      
\subsection{B$^+$-Tree and Prefix B$^+$-Tree}

\subsection{Log-Structured Merge-Tree (LSM-Tree)}

\subsection{Couchstore}

\newpage

\section{FORESTDB DESIGN}

\subsection{HB$^+$-Trie}

\subsection{Optimizations for Avoiding Skew in HB$^+$-Trie}

\subsubsection{Overview}

\subsubsection{Leaf B$^+$-Tree Extension}

\subsubsection{Extension Threshold Analysis}

\subsection{Log-Structured Write Buffer}

\newpage

\section{EVALUATION}

\subsection{Comparison of Index Structures}

\subsection{Full System Performance}

\subsubsection{Key Length}

\subsubsection{Read/Write Ratio}

\subsubsection{Concurrent Read/Write Operation}

\subsubsection{Range Scan}

\subsubsection{The Effect of Write Buffer and HB$^+$-Trie}

\subsubsection{Locality}

\subsubsection{Real Dataset Results}

\newpage

\section{RELATED WORK}

\newpage

\section{CONCLUSIONS AND FUTURE WORK}

\newpage

\section{ACKNOWLEDGMENTS}

\newpage

\section{参考文献}
\begin{flushleft}
[1] Couchbase NoSQL Database. [Online]. Available: http://www.couchbase.com/, 2011.

[2] J. Dean and S. Ghemawat. (2011). LevelDB: A fast and lightweight key/value database library by Google. [Online]. Available: https://github.com/google/leveldb

[3] RocksDB: a persistent key-value store for fast storage environments. [Online]. Available: http://rocksdb.org/, 2013.

[4] J. Han, E. Haihong, G. Le, and J. Du, "Survey on nosql database," in Proc. IEEE 6th Int. Conf. Pervasive Computing and Applications (ICPSCA), 2011, pp. 363–366.

[5] D. Comer,"Ubiquitous B-tree," ACM Comput. Surveys, vol. 11, no. 2, pp. 121–137, 1979.

[6] P. O’Neil, E. Cheng, D. Gawlick, andE. O’Neil, "The log-structured merge-tree (LSM-tree)," Acta Informatica, vol. 33, no. 4, pp. 351–385, 1996.

[7] M. A. Olson, K. Bostic, and M. I. Seltzer, "Berkeley DB," in Proc. USENIX Annu. Tech. Conf., FREENIX Track, 1999, pp. 183–191.

[8] P. Fruhwirt, M. Huber, M. Mulazzani, and E. R. Weippl, "InnoDB database forensics," in Proc. 24th IEEE Int. Conf. Adv. Inf. Netw. Appl., 2010, pp. 1028–1036.

[9] MongoDB. [Online]. Available: http://www.mongodb.com, 2009.

[10] SQLite4. [Online]. Available: https://sqlite.org/src4/doc/trunk/www/index.wiki, 2013.

[11] A. Lakshman and P. Malik, "Cassandra: A decentralized structured storage system," ACM SIGOPS Oper. Syst. Rev., vol. 44, no. 2, pp. 35–40, 2010.

[12] F. Chang, J. Dean, S. Ghemawat, W. C. Hsieh, D. A. Wallach, M. Burrows, T. Chandra, A. Fikes, and R. E. Gruber, "Bigtable: A distributed storage system for structured data," ACM Trans. Comput. Syst., vol. 26, no. 2, p. 4, 2008.

[13] R. Bayer and K. Unterauer, "Prefix B-trees," ACM Trans. Database Syst., vol. 2, no. 1, pp. 11–26, 1977.

[14] A. A. Moffat, T. C. Bell, and I. H. Witten, Managing Gigabytes: Compressing and Indexing Documents and Images. San Mateo, CA, USA: Morgan Kaufmann, 1999.

[15] D. R. Morrison, "Patricia: Practical algorithm to retrieve information coded in alphanumeric," J. ACM, vol. 15, no. 4, pp. 514–534, 1968.

[16] P. A. Bernstein and N. Goodman, "Concurrency control in distributed database systems," ACM Comput. Surveys, vol. 13, no. 2, pp. 185–221, 1981.

[17] J. C. Anderson, J. Lehnardt, and N. Slater, CouchDB: The Definitive Guide. O’Reilly Media, Inc., Sebastopol, CA, 2010.

[18] Apache CouchDB. [Online]. Available: http://couchdb.apache.org/, 2005.

[19] J.-S. Ahn. (2011). Third annual SIGMOD programming contest. [Online]. Available: http://goo.gl/yKUoiY

[20] Snappy: A fast compressor/decompressor. [Online]. Available: http://code.google.com/p/snappy/, 2011.

[21] G. K. Zipf, Human Behavior and the Principle of Least Effort, Cambridge, MA, Addison-Wesley Press, 1949.

[22] "WEBSPAM-UK2007. [Online]. Available: http://chato.cl/webspam/datasets/uk2007/.

[23] Y. Mao, E. Kohler, and R. T. Morris, "Cache craftiness for fast multicore key-value storage," in Proc. 7th ACM Eur. Conf. Comput. Syst., 2012, pp. 183–196.

[24] H. Lim, B. Fan, D. G. Andersen, and M. Kaminsky, "SILT: A memory-efficient, high-performance key-value store," in Proc. 23rd ACM Symp. Oper. Syst. Principles., 2011, pp. 1–13.

[25] B. Debnath, S. Sengupta, and J. Li, "FlashStore: High throughput persistent key-value store," in Proc. VLDB Endowment, vol. 3, no. 1-2, pp. 1414–1425, 2010.

[26] B. Debnath, S. Sengupta, and J. Li, "SkimpyStash: RAM space skimpy key-value store on flash-based storage," in Proc. ACM SIGMOD Int. Conf. Manage. Data., 2011, pp. 25–36.

[27] J. Gray, "The transaction concept: Virtues and limitations," in Proc. 7th Int. Conf. Very Large Data Bases, 1981, vol. 81, pp. 144–154.

[28] Riak. [Online]. Available: http://basho.com/riak/, 2009.

[29] G. DeCandia, D. Hastorun, M. Jampani, G. Kakulapati, A. Lakshman, A. Pilchin, S. Sivasubramanian, P. Vosshall, and W. Vogels, "Dynamo: Amazon’s highly available key-value store," ACM SIGOPS Oper. Syst. Rev., vol. 41, no. 6, pp. 205–220, 2007.

[30] R. Sears and R. Ramakrishnan, "bLSM: A general purpose log structured merge tree," in Proc. ACM SIGMOD Int. Conf. Manage. Data., 2012, pp. 217–228.

[31] B. F. Cooper, R. Ramakrishnan, U. Srivastava, A. Silberstein, P. Bohannon, H.-A. Jacobsen, N. Puz, D. Weaver, and R. Yerneni, "PNUTS: Yahoo!’s hosted data serving platform," in Proc. VLDB Endowment, 2008, vol. 1, no. 2, pp. 1277–1288.

[32] J. Chen, C. Douglas, M. Mutsuzaki, P. Quaid, R. Ramakrishnan, S. Rao, and R. Sears, "Walnut: A unified cloud object store," in Proc. ACM SIGMOD Int. Conf. Manage. Data., 2012, pp. 743–754.

\end{flushleft}

\end{document}
