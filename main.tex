%!TEX program = xelatex
\documentclass[lang=cn]{cls/elegantpaper}

\title{ForestDB: A Fast Key-Value Storage System for Variable-Length String Keys}
\author{Jung-Sang Ahn, Chiyoung Seo, Ravi Mayuram, Rahim Yaseen, Jin-Soo Kim, and Seungryoul Maeng}

\begin{document}

\maketitle

\begin{abstract}
\noindent \sffamily 持久化存储的键值映射数据对NoSQL数据库而言是非常重要的组成部分。大多数键值存储引擎使用树型结构建立数据索引,但是他们的时间和空间开销随着键长度的增加而急剧增大。这同时也影响着对整体吞吐量至关重要的合并和压缩操作的开销。这篇论文介绍了一个单节点大规模NoSQL数据库存储引擎,我们将其命名为ForestDB。ForestDB使用HB$^+$字典树索引方案,这是一种基于磁盘的结合字典树与B$^+$树的新型混合索引结构。对比于其他树型结构,它索引性能较高。即便是键非常长,在空间内随机分布,HB$^+$字典树也可以以较低的磁盘访问量获取任意长度的键。我们的评测结果显示,在每秒操作数与每更新操作磁盘写入数两个性能指标上,ForestDB都显著的优于Couchbase Server、LevelDB和RocksDB现有的键值存储引擎。\\

\noindent \sffamily {\bfseries 关键词\quad} 键值存储系统\quad NoSQL\quad 索引结构\quad B$^+$树
\end{abstract}
\section{介绍}

近年来,数据管理和索引的需求愈发强烈。在诸如Facebook和Twitter这样的社交网络中,并发用户量和数据处理量越来越大,而数据本身也逐步趋向于非结构化以灵活的被应用分析。不幸的是,受限于可扩展性与严格的数据模型要求,关系型数据库难以解决这些问题。因此许多公司使用NoSQL作为关系型数据库的替代品。

虽然现有许多关于高级分片、数据复制和分布式缓存的NoSQL技术,但他们底层的持久化存储模块差别不大。单节点NoSQL通常使用键值存储来实现索引或获取无模式数据,其键与值通常是不定长的字符串。\footnote{在某些键值存储中,值可能是半结构化的数据(例如JSON文档),但它仍然可被看做字符串。}因为键值存储直接作用于磁盘(HHD)或固态硬盘(SSD)这样的通用块设备,其吞吐量和延迟决定了系统的整体性能。

键值存储的吞吐量和延迟受限于存储设备访问时间,其主要受两个因素影响:每次键值操作要访问的块数量及块访问模式。前者主要与索引结构的特点与逻辑设计有关,后者取决于含键值对在内的索引数据如何真正的写入或读取存储设备。

有两种广泛用于单机键值存储的索引结构:B$^+$树和结构化日志合并树(LSM树)。其中B$^+$树是最普及的索引结构之一,得益于它减少I/O操作的能力,它广泛应用于传统数据库。现代键值数据库(例如BerkeleyDB、Couchbase、InnoDB和MongoDB)使用B$^+$树作为底层存储。相反,LSM树由一组分层B$^+$树(或类似B$^+$树结构)组成,相较于传统B$^+$树,它通过牺牲读性能来提高写性能。很多近代系统(例如LevelDB、RocksDB、SQLite4、Cassandra和BigTable)使用LSM树或者其变种构建键值索引。

虽然目前为止,这些树型结构取得了成功,但是当它们使用键是变长字符串而非定长原始类型时,它们的性能将下降。如果节点尺寸是固定的,随着键尺寸的增大,节点扇出(例如节点中键指针数)将减少,为了保持同样的容量,树的高度将增大。另一方面,如果为了维持扇出数而增加节点尺寸,访问节点需要读写的块数量将会同比增加。不幸的是,树结构的高度和节点尺寸直接影响平均磁盘访问量和空间占用量,因此随着键长度的增加,索引的整体性能将下降。

为了解决这个问题,BerkeleyDB和LevelDB使用一种类似于前缀B$^+$树和预编码的前缀压缩技术。然而,这种方案大程度受限于键的模式。如果键在键空间内随机分布,键中剩余的未压缩部分仍然过长,因此前缀压缩技术的收益十分有限。为了索引变长字符串键,我们需要设计一种更有效的方法。

与此同时,在键值存储系统的设计中,块访问模式是另一个重要的因素。无论是HDD还是SSD,在操作块数相同的情况下,块读写顺序的分布很大程度的影响I/O性能。基于就地更新的策略的确能达到一个优秀的读性能,但是往往随之而来的是糟糕的写入延迟,对于近代写密集型场景来说,这是不可接受的。因此大多数数据库使用只追加或先行写入日志(WAL)的方式顺序的写入块。

这样的设计可以达到高写入吞吐量,但要付出合并和压缩(例如,垃圾收集)的开销。此类开销与平均每索引操作块访问数和索引结构空间占用量密切相关,因为在压缩过程中会执行许多合并操作。于是当长的键引起索引结构扇出降低时,开销将变大。

本文介绍了一个用于下一代Couchbase Server的单节点键值存储系统——ForestDB。为了在键尺寸较大和键随机分布的情况下均能在时间和空间角度高效的索引变长字符串键,我们提出了一种新颖的基于磁盘的索引结构——分层B$^+$字典树(HB$^+$字典树),作为ForestDB中的主键。HB$^+$字典树的逻辑结构基本上是Patricia字典树的一个变种,但它使用B$^+$树降低持久化存储设备上的块访问量。为了使读写操作都达到一个较高的吞吐量,并且支持高并发访问,更新索引操作使用只追加的方式写入存储设备。这也简化了磁盘结构以实现多版本并发控制(MVCC)。

我们以单机键值存储库的方式实现了ForestDB。我们使用真实数据集进行评测,结果显示ForestDB的平均每秒操作数显著高于LevelDB、RocksDB、Couchstore和Couchbase Server现有的键值存储模块。本文的其余部分安排如下。第二章概述了B$^+$树、前缀B$^+$树、LSM树和当前Couchstore的内部结构。第三章介绍了ForestDB的总体设计。第四章介绍评测结果,第五章介绍了相关工作。第六章总结全文。
\section{背景}
\subsection{B$^+$树和前缀B$^+$树}

B$^+$树是块设备上最广泛的管理海量记录的索引结构之一。B$^+$树中有两种不同类型的节点:叶节点和索引(非叶)节点。不同于B树或其他二叉搜索树,键值记录只储存于叶节点,索引节点则包含子节点指针。在B$^+$树中,每个节点不只包含两个键值对(或键指针);每结点中键值对的数量称作扇出。一般来说,为了使B$^+$树节点与一个或多个块对齐,B$^+$树的扇出会被设置为一个比较大的值。B$^+$树以这种方式降低每访问键值对产生的块I/O访问数。

然而,正如我们之前提到的,键长度对扇出影响极大。为了缓和这方面开销,前缀B$^+$树被提出了。前缀B$^+$树的主要思想是只存储可辨识的部分子串而不是整个字符串,通过这种方式节省存储空间,提高扇出。索引节点只储存最小的可区分其子节点的前缀,而叶节点则跳过同节点中相同的前缀。

\begin{figure}[htbp]
    \centering
    \begin{overpic}[scale=0.5]{bptree_and_prefix_bptree.png}
        \put(53,-3){\scriptsize 公共前缀:b}
        \put(86.5,-3){\scriptsize 公共前缀:d}
        \put(17,-7){\scriptsize (a)传统B$^+$树}
        \put(69,-7){\scriptsize (b)前缀B$^+$树}
    \end{overpic}
    \\[3em]
	\caption{B$^+$树及前缀B$^+$树示例\label{fig:bptree_and_prefix_bptree}}
\end{figure}

图\ref{fig:bptree_and_prefix_bptree}展示了一个B$^+$树和相应的前缀B$^+$树的例子。在前缀B$^+$树中,根(即索引节点)节点只存储字符串$c$而非整个键$car$,因为它是字典序大于$band$、$bingo$和$black$且小于等于$car$、$chalk$和$diary$的最小子串。但我们不能使用$d$代替$dim$因为$d$不能区分$diary$和$dime$。在本例中,前缀B$^+$树使用$dim$,因为它是字典序介于$diary$和$dime$的最小子串。

在叶节点中,最左边节点中的$band$、$bingo$和$black$有公共前缀$b$,最右边节点中的$dime$和$diary$和$dog$有公共前缀$d$。前缀B$^+$树在每个节点中的元数据部分记录这些公共前缀,只存储这些字符串的非公有部分。当大量的键有公共前缀时,这种前缀压缩方案有效的降低了空间开销。

\subsection{结构化日志合并树(LSM树)}

尽管B$^+$树可以使索引大量记录时产生的块访问量最小化,但由于随机块访问,依然可能会遇到性能不佳的问题。随时间推移,B$^+$树的节点将随机散落在硬盘上。这将在树遍历时产生块随机访问,从而引起性能大幅度降低。得益于B$^+$树最小化块访问的能力,几乎没有其他树型结构的随机读取性能比B$^+$树好,但我们仍然可以通过整理磁盘写入操作的思路提高其写入性能。

\begin{figure}[htbp]
    \centering
    \begin{overpic}[scale=1]{lsmtree.png}
        \put(9,18){\scriptsize C$_0$树}
        \put(39,5){\scriptsize C$_1$树}
        \put(74,5){\scriptsize C$_2$树}
        \put(7,5){\scriptsize 顺序日志}
        \put(25,22){\scriptsize 刷新}
        \put(53,22){\scriptsize 刷新}
        \put(9.5,27){\scriptsize 内存}
        \put(38,-2.5){\scriptsize 容量呈指数增长}
    \end{overpic}
    \\[1.5em]
	\caption{B$^+$树及前缀B$^+$树示例\label{fig:lsmtree}}
\end{figure}

LSM树是一种超越B$^+$树的随机写入性能的索引结构。如图\ref{fig:lsmtree}所示,一些B$^+$树或类B$^+$树结构组织一个单独的LSM树。在LSM树上层有一个存储于内存的树结构,被称为C$_0$树。所有更新记录追加至一个顺序日志中,C$_0$树索引着日志中的每一个项以便高效访问。一旦C$_0$树的尺寸超过了某个阈值,一段连续的日志记录将被合并进C$_1$树。当C$_1$树的尺寸超过某个阈值,C$_1$树中一段连续的记录会以同样的方式合并进C$_2$树。合并操作总是在C$_i$树和C$_{i+1}$树之间发生,通常树的容量随$i$指数增长。

因为追加日志和合并操作都是对磁盘的顺序写入,因此LSM树的写入性能要比传统B$^+$树好得多。然而要访问一个键值对,我们只能从C$_0$开始对所有的树进行级联查找,直到发现目标记录,因此LSM的读取开销较B$^+$树大。为了避免不必要的遍历操作,LSM树一般使用布隆过滤器减弱读放大效应。

请注意第1章中提出的问题在LSM树中同样存在。如果键长度变长,较小的扇出会降低每棵树的容量,树间的合并操作将更加频繁。因为合并会触发大量I/O操作,导致性能的急剧下降。

\subsection{Couchstore}

\begin{figure}[htbp]
    \centering
    ~\\
    \begin{overpic}[scale=1]{couchstore.png}
        \put(55,2){\tiny 字节偏移量}
        \put(75,49){\scriptsize 最新数据}
        \put(75,42){\scriptsize 陈旧数据}
        \put(75,34){\scriptsize B$^+$树节点}
        \put(75,28){\scriptsize 文档}
        \put(66,11){\scriptsize 只追加}
        \put(4,19){\scriptsize DB文件的非结构化视图}
        \put(4,55){\scriptsize B$^+$树节点和文档的逻辑视图}
        \put(62,61){\scriptsize 单一Couchstore实例}
        \put(41,95){\scriptsize 键空间}
    \end{overpic}
	\caption{Couchstore概览\label{fig:couchstore}}
\end{figure}

Couchstore 是Couchbase中的单节点存储引擎,其整体架构继承自Apache CouchDB的存储模块。键空间被等分为若干用户定义的键范围数,我们称其为vBucket(或分区),每个vBucket拥有它自己的DB文件。每个DB文件所存储的键值对属于其对应的vBucket,每个键是一个任意长的字符串,每个值是JSON文档。为了确定文件中某个文档的位置,每个vBucket中有一个B$^+$树用于存储键和其对应文档的字节偏移量所构成的元组。因此,每一个DB文件包含文档和B$^+$树节点两部分,他们互相交错存储于文件中。

如图\ref{fig:couchstore}所示,在Couchstore中所有的更新操作都追加在DB文件的后面。$A$、$B$和$C$代表B$^+$树节点,在胶囊图形中的$D1$、$D2$和$D3$代表文档。如果$D1$文档更新了,新文档$D1'$将被写入文件末尾,而不会擦除或修改原文档$D1$。因为文档位置更新了,$B$节点必须更新为$B'$节点,同样追加至文件末尾。更新操作一直传递到根节点$A$,最终新根节点$A'$被写入到$B'$节点后。

相比于就地更新策略,只追加的B$^+$树可以实现非常高的写入吞吐量,因为所有的磁盘写入操作都是顺序的。此外,我们不需要牺牲读取性能,因为读取过程与传统B$^+$树是相同的。但是DB文件的占用空间随更新操作而增长,因此我们必须定期回收陈旧数据所占用的空间。当陈旧数据与总数据尺寸比超过一定预设阈值时,Couchstore触发器将执行这个压缩过程。当前目标DB文件中所有的文档将被移动到新DB文件,在压缩操作完成后,旧DB文件将被删除。在压缩过程中,所有对目标文件的写入操作将被阻塞,而读取操作是被允许的。每次仅对一个DB文件进行压缩。

与传统B$^+$树处理字符串的方式相同,为了维持容量不变,如果键长度增加,树的高度将增加。因为追加数据量与树高度成正比,因此这种只追加设计的性能衰减可能会更糟糕。由此导致压缩操作被更频繁的触发,整体性能陷入恶性循环。我们需要为变长字符串键设计一个更紧凑更有效的索引结构。
\section{FORESTDB DESIGN}

\subsection{HB$^+$-Trie}

\subsection{Optimizations for Avoiding Skew in HB$^+$-Trie}

\subsubsection{Overview}

\subsubsection{Leaf B$^+$-Tree Extension}

\subsubsection{Extension Threshold Analysis}

\subsection{Log-Structured Write Buffer}

\section{EVALUATION}

\subsection{Comparison of Index Structures}

\subsection{Full System Performance}

\subsubsection{Key Length}

\subsubsection{Read/Write Ratio}

\subsubsection{Concurrent Read/Write Operation}

\subsubsection{Range Scan}

\subsubsection{The Effect of Write Buffer and HB$^+$-Trie}

\subsubsection{Locality}

\subsubsection{Real Dataset Results}

\section{RELATED WORK}

\section{CONCLUSIONS AND FUTURE WORK}

\section{ACKNOWLEDGMENTS}

\section{参考文献}
\begin{flushleft}
[1] Couchbase NoSQL Database. [Online]. Available: http://www.couchbase.com/, 2011.

[2] J. Dean and S. Ghemawat. (2011). LevelDB: A fast and lightweight key/value database library by Google. [Online]. Available: https://github.com/google/leveldb

[3] RocksDB: a persistent key-value store for fast storage environments. [Online]. Available: http://rocksdb.org/, 2013.

[4] J. Han, E. Haihong, G. Le, and J. Du, "Survey on nosql database," in Proc. IEEE 6th Int. Conf. Pervasive Computing and Applications (ICPSCA), 2011, pp. 363–366.

[5] D. Comer,"Ubiquitous B-tree," ACM Comput. Surveys, vol. 11, no. 2, pp. 121–137, 1979.

[6] P. O’Neil, E. Cheng, D. Gawlick, andE. O’Neil, "The log-structured merge-tree (LSM-tree)," Acta Informatica, vol. 33, no. 4, pp. 351–385, 1996.

[7] M. A. Olson, K. Bostic, and M. I. Seltzer, "Berkeley DB," in Proc. USENIX Annu. Tech. Conf., FREENIX Track, 1999, pp. 183–191.

[8] P. Fruhwirt, M. Huber, M. Mulazzani, and E. R. Weippl, "InnoDB database forensics," in Proc. 24th IEEE Int. Conf. Adv. Inf. Netw. Appl., 2010, pp. 1028–1036.

[9] MongoDB. [Online]. Available: http://www.mongodb.com, 2009.

[10] SQLite4. [Online]. Available: https://sqlite.org/src4/doc/trunk/www/index.wiki, 2013.

[11] A. Lakshman and P. Malik, "Cassandra: A decentralized structured storage system," ACM SIGOPS Oper. Syst. Rev., vol. 44, no. 2, pp. 35–40, 2010.

[12] F. Chang, J. Dean, S. Ghemawat, W. C. Hsieh, D. A. Wallach, M. Burrows, T. Chandra, A. Fikes, and R. E. Gruber, "Bigtable: A distributed storage system for structured data," ACM Trans. Comput. Syst., vol. 26, no. 2, p. 4, 2008.

[13] R. Bayer and K. Unterauer, "Prefix B-trees," ACM Trans. Database Syst., vol. 2, no. 1, pp. 11–26, 1977.

[14] A. A. Moffat, T. C. Bell, and I. H. Witten, Managing Gigabytes: Compressing and Indexing Documents and Images. San Mateo, CA, USA: Morgan Kaufmann, 1999.

[15] D. R. Morrison, "Patricia: Practical algorithm to retrieve information coded in alphanumeric," J. ACM, vol. 15, no. 4, pp. 514–534, 1968.

[16] P. A. Bernstein and N. Goodman, "Concurrency control in distributed database systems," ACM Comput. Surveys, vol. 13, no. 2, pp. 185–221, 1981.

[17] J. C. Anderson, J. Lehnardt, and N. Slater, CouchDB: The Definitive Guide. O’Reilly Media, Inc., Sebastopol, CA, 2010.

[18] Apache CouchDB. [Online]. Available: http://couchdb.apache.org/, 2005.

[19] J.-S. Ahn. (2011). Third annual SIGMOD programming contest. [Online]. Available: http://goo.gl/yKUoiY

[20] Snappy: A fast compressor/decompressor. [Online]. Available: http://code.google.com/p/snappy/, 2011.

[21] G. K. Zipf, Human Behavior and the Principle of Least Effort, Cambridge, MA, Addison-Wesley Press, 1949.

[22] "WEBSPAM-UK2007. [Online]. Available: http://chato.cl/webspam/datasets/uk2007/.

[23] Y. Mao, E. Kohler, and R. T. Morris, "Cache craftiness for fast multicore key-value storage," in Proc. 7th ACM Eur. Conf. Comput. Syst., 2012, pp. 183–196.

[24] H. Lim, B. Fan, D. G. Andersen, and M. Kaminsky, "SILT: A memory-efficient, high-performance key-value store," in Proc. 23rd ACM Symp. Oper. Syst. Principles., 2011, pp. 1–13.

[25] B. Debnath, S. Sengupta, and J. Li, "FlashStore: High throughput persistent key-value store," in Proc. VLDB Endowment, vol. 3, no. 1-2, pp. 1414–1425, 2010.

[26] B. Debnath, S. Sengupta, and J. Li, "SkimpyStash: RAM space skimpy key-value store on flash-based storage," in Proc. ACM SIGMOD Int. Conf. Manage. Data., 2011, pp. 25–36.

[27] J. Gray, "The transaction concept: Virtues and limitations," in Proc. 7th Int. Conf. Very Large Data Bases, 1981, vol. 81, pp. 144–154.

[28] Riak. [Online]. Available: http://basho.com/riak/, 2009.

[29] G. DeCandia, D. Hastorun, M. Jampani, G. Kakulapati, A. Lakshman, A. Pilchin, S. Sivasubramanian, P. Vosshall, and W. Vogels, "Dynamo: Amazon’s highly available key-value store," ACM SIGOPS Oper. Syst. Rev., vol. 41, no. 6, pp. 205–220, 2007.

[30] R. Sears and R. Ramakrishnan, "bLSM: A general purpose log structured merge tree," in Proc. ACM SIGMOD Int. Conf. Manage. Data., 2012, pp. 217–228.

[31] B. F. Cooper, R. Ramakrishnan, U. Srivastava, A. Silberstein, P. Bohannon, H.-A. Jacobsen, N. Puz, D. Weaver, and R. Yerneni, "PNUTS: Yahoo!’s hosted data serving platform," in Proc. VLDB Endowment, 2008, vol. 1, no. 2, pp. 1277–1288.

[32] J. Chen, C. Douglas, M. Mutsuzaki, P. Quaid, R. Ramakrishnan, S. Rao, and R. Sears, "Walnut: A unified cloud object store," in Proc. ACM SIGMOD Int. Conf. Manage. Data., 2012, pp. 743–754.

\end{flushleft}

\end{document}
