%!TEX program = xelatex
\documentclass[lang=cn]{elegantpaper}

\title{ForestDB: A Fast Key-Value Storage System for Variable-Length String Keys}
\author{Jung-Sang Ahn, Chiyoung Seo, Ravi Mayuram, Rahim Yaseen, Jin-Soo Kim, and Seungryoul Maeng}

\begin{document}

\maketitle

\begin{abstract}
\noindent \sffamily 基于持久存储的键值映射数据对NoSQL数据库而言是非常重要的组成部分。大多数键值存储引擎使用树型结构建立数据索引,但是他们的时间和空间开销随着键长度的增加而急剧增大。这同时也影响着对整体吞吐量至关重要的合并和压缩操作的开销。这篇论文介绍了一个单节点大规模NoSQL数据库存储引擎,我们将其命名为ForestDB。ForestDB使用HB$^+$字典树索引方案,这是一种基于磁盘的结合字典树与B$^+$树的新型混合索引结构。对比于其他树型结构,它索引性能较高。即便是键非常长,在空间内随机分布,HB$^+$字典树也可以以较低的磁盘访问量获取任意长度的键。我们的评测结果显示,在每秒操作数与每更新操作磁盘写入数两个性能指标上,ForestDB都显著的优于Couchbase Server、LevelDB和RocksDB现有的键值存储引擎。\\

\noindent \sffamily {\bfseries 关键词\quad} 键值存储系统\quad NoSQL\quad 索引结构\quad B$^+$树
\end{abstract}

\section{INTRODUCTION}

\newpage

\section{BACKGROUND}
      
\subsection{B$^+$-Tree and Prefix B$^+$-Tree}

\subsection{Log-Structured Merge-Tree (LSM-Tree)}

\subsection{Couchstore}

\newpage

\section{FORESTDB DESIGN}

\subsection{HB$^+$-Trie}

\subsection{Optimizations for Avoiding Skew in HB$^+$-Trie}

\subsubsection{Overview}

\subsubsection{Leaf B$^+$-Tree Extension}

\subsubsection{Extension Threshold Analysis}

\subsection{Log-Structured Write Buffer}

\newpage

\section{EVALUATION}

\subsection{Comparison of Index Structures}

\subsection{Full System Performance}

\subsubsection{Key Length}

\subsubsection{Read/Write Ratio}

\subsubsection{Concurrent Read/Write Operation}

\subsubsection{Range Scan}

\subsubsection{The Effect of Write Buffer and HB$^+$-Trie}

\subsubsection{Locality}

\subsubsection{Real Dataset Results}

\newpage

\section{RELATED WORK}

\newpage

\section{CONCLUSIONS AND FUTURE WORK}

\newpage

\section{ACKNOWLEDGMENTS}

\newpage

\section{REFERENCES}

\end{document}
