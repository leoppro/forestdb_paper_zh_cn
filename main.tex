%!TEX program = xelatex
\documentclass[lang=cn]{cls/elegantpaper}

\hypersetup{pdfauthor={Jung-Sang Ahn等,译者:赵一霖},
            pdftitle={ForestDB:为不定⻓字符串键设计的高速键值存储系统},
            pdfsubject={ForestDB:为不定⻓字符串键设计的高速键值存储系统},
            pdfkeywords={键值存储系统、NoSQL、索引结构、B+树},
            pdfproducer={LaTeX},
            pdfcreator={XeLatex}
}

\title{ForestDB:为不定长字符串键设计的高速键值存储系统}
\author{Jung-Sang Ahn, Chiyoung Seo, Ravi Mayuram, Rahim Yaseen, Jin-Soo Kim, and Seungryoul Maeng}

\begin{document}

\maketitle

\begin{abstract}
\noindent \sffamily 持久化存储的键值映射数据对NoSQL数据库而言是非常重要的组成部分。大多数键值存储引擎使用树型结构建立数据索引,但是他们的时间和空间开销随着键长度的增加而急剧增大。这同时也影响着对整体吞吐量至关重要的合并和压缩操作的开销。这篇论文介绍了一个单节点大规模NoSQL数据库存储引擎,我们将其命名为ForestDB。ForestDB使用HB$^+$字典树索引方案,这是一种基于磁盘的结合字典树与B$^+$树的新型混合索引结构。对比于其他树型结构,它索引性能较高。即便是键非常长,在空间内随机分布,HB$^+$字典树也可以以较低的磁盘访问量获取任意长度的键。我们的评测结果显示,在每秒操作数与每更新操作磁盘写入数两个性能指标上,ForestDB都显著的优于Couchbase Server、LevelDB和RocksDB现有的键值存储引擎。\\

\noindent \sffamily {\bfseries 关键词\quad} 键值存储系统\quad NoSQL\quad 索引结构\quad B$^+$树
\end{abstract}

\section{介绍}

近年来,数据管理和索引的需求愈发强烈。在诸如Facebook和Twitter这样的社交网络中,并发用户量和数据处理量越来越大,而数据本身也逐步趋向于非结构化以灵活的被应用分析。不幸的是,受限于可扩展性与严格的数据模型要求,关系型数据库难以解决这些问题。因此许多公司使用NoSQL作为关系型数据库的替代品。

虽然现有许多关于高级分片、数据复制和分布式缓存的NoSQL技术,但他们底层的持久化存储模块差别不大。单节点NoSQL通常使用键值存储来实现索引或获取无模式数据,其键与值通常是不定长的字符串。\footnote{在某些键值存储中,值可能是半结构化的数据(例如JSON文档),但它仍然可被看做字符串。}因为键值存储直接作用于磁盘(HHD)或固态硬盘(SSD)这样的通用块设备,其吞吐量和延迟决定了系统的整体性能。

键值存储的吞吐量和延迟受限于存储设备访问时间,其主要受两个因素影响:每次键值操作要访问的块数量及块访问模式。前者主要与索引结构的特点与逻辑设计有关,后者取决于含键值对在内的索引数据如何真正的写入或读取存储设备。

有两种广泛用于单机键值存储的索引结构:B$^+$树和结构化日志合并树(LSM树)。其中B$^+$树是最普及的索引结构之一,得益于它减少I/O操作的能力,它广泛应用于传统数据库。现代键值数据库(例如BerkeleyDB、Couchbase、InnoDB和MongoDB)使用B$^+$树作为底层存储。相反,LSM树由一组分层B$^+$树(或类似B$^+$树结构)组成,相较于传统B$^+$树,它通过牺牲读性能来提高写性能。很多近代系统(例如LevelDB、RocksDB、SQLite4、Cassandra和BigTable)使用LSM树或者其变种构建键值索引。

虽然目前为止,这些树型结构取得了成功,但是当它们使用键是变长字符串而非定长原始类型时,它们的性能将下降。如果节点尺寸是固定的,随着键尺寸的增大,节点扇出(例如节点中键指针数)将减少,为了保持同样的容量,树的高度将增大。另一方面,如果为了维持扇出数而增加节点尺寸,访问节点需要读写的块数量将会同比增加。不幸的是,树结构的高度和节点尺寸直接影响平均磁盘访问量和空间占用量,因此随着键长度的增加,索引的整体性能将下降。

为了解决这个问题,BerkeleyDB和LevelDB使用一种类似于前缀B$^+$树和预编码的前缀压缩技术。然而,这种方案大程度受限于键的模式。如果键在键空间内随机分布,键中剩余的未压缩部分仍然过长,因此前缀压缩技术的收益十分有限。为了索引变长字符串键,我们需要设计一种更有效的方法。

与此同时,在键值存储系统的设计中,块访问模式是另一个重要的因素。无论是HDD还是SSD,在操作块数相同的情况下,块读写顺序的分布很大程度的影响I/O性能。基于就地更新的策略的确能达到一个优秀的读性能,但是往往随之而来的是糟糕的写入延迟,对于近代写密集型场景来说,这是不可接受的。因此大多数数据库使用只追加或先行写入日志(WAL)的方式顺序的写入块。

这样的设计可以达到高写入吞吐量,但要付出合并和压缩(例如,垃圾收集)的开销。此类开销与平均每索引操作块访问数和索引结构空间占用量密切相关,因为在压缩过程中会执行许多合并操作。于是当长的键引起索引结构扇出降低时,开销将变大。

本文介绍了一个用于下一代Couchbase Server的单节点键值存储系统——ForestDB。为了在键尺寸较大和键随机分布的情况下均能在时间和空间角度高效的索引变长字符串键,我们提出了一种新颖的基于磁盘的索引结构——分层B$^+$字典树(HB$^+$字典树),作为ForestDB中的主键。HB$^+$字典树的逻辑结构基本上是Patricia字典树的一个变种,但它使用B$^+$树降低持久化存储设备上的块访问量。为了使读写操作都达到一个较高的吞吐量,并且支持高并发访问,更新索引操作使用只追加的方式写入存储设备。这也简化了磁盘结构以实现多版本并发控制(MVCC)。

我们以单机键值存储库的方式实现了ForestDB。我们使用真实数据集进行评测,结果显示ForestDB的平均每秒操作数显著高于LevelDB、RocksDB、Couchstore和Couchbase Server现有的键值存储模块。本文的其余部分安排如下。第二章概述了B$^+$树、前缀B$^+$树、LSM树和当前Couchstore的内部结构。第三章介绍了ForestDB的总体设计。第四章介绍评测结果,第五章介绍了相关工作。第六章总结全文。
\section{背景}
\subsection{B$^+$树和前缀B$^+$树}

B$^+$树是块设备上最广泛的管理海量记录的索引结构之一。B$^+$树中有两种不同类型的节点:叶节点和索引(非叶)节点。不同于B树或其他二叉搜索树,键值记录只储存于叶节点,索引节点则包含子节点指针。在B$^+$树中,每个节点不只包含两个键值对(或键指针);每结点中键值对的数量称作扇出。一般来说,为了使B$^+$树节点与一个或多个块对齐,B$^+$树的扇出会被设置为一个比较大的值。B$^+$树以这种方式降低每访问键值对产生的块I/O访问数。

然而,正如我们之前提到的,键长度对扇出影响极大。为了缓和这方面开销,前缀B$^+$树被提出了。前缀B$^+$树的主要思想是只存储可辨识的部分子串而不是整个字符串,通过这种方式节省存储空间,提高扇出。索引节点只储存最小的可区分其子节点的前缀,而叶节点则跳过同节点中相同的前缀。

\begin{figure}[htbp]
    \centering
    \begin{overpic}[scale=0.5]{bptree_and_prefix_bptree.png}
        \put(53,-3){\scriptsize 公共前缀:b}
        \put(86.5,-3){\scriptsize 公共前缀:d}
        \put(17,-7){\scriptsize (a)传统B$^+$树}
        \put(69,-7){\scriptsize (b)前缀B$^+$树}
    \end{overpic}
    \\[3em]
	\caption{B$^+$树及前缀B$^+$树示例\label{fig:bptree_and_prefix_bptree}}
\end{figure}

图\ref{fig:bptree_and_prefix_bptree}展示了一个B$^+$树和相应的前缀B$^+$树的例子。在前缀B$^+$树中,根(即索引节点)节点只存储字符串$c$而非整个键$car$,因为它是字典序大于$band$、$bingo$和$black$且小于等于$car$、$chalk$和$diary$的最小子串。但我们不能使用$d$代替$dim$因为$d$不能区分$diary$和$dime$。在本例中,前缀B$^+$树使用$dim$,因为它是字典序介于$diary$和$dime$的最小子串。

在叶节点中,最左边节点中的$band$、$bingo$和$black$有公共前缀$b$,最右边节点中的$dime$和$diary$和$dog$有公共前缀$d$。前缀B$^+$树在每个节点中的元数据部分记录这些公共前缀,只存储这些字符串的非公有部分。当大量的键有公共前缀时,这种前缀压缩方案有效的降低了空间开销。

\subsection{结构化日志合并树(LSM树)}

尽管B$^+$树可以使索引大量记录时产生的块访问量最小化,但由于随机块访问,依然可能会遇到性能不佳的问题。随时间推移,B$^+$树的节点将随机散落在硬盘上。这将在树遍历时产生块随机访问,从而引起性能大幅度降低。得益于B$^+$树最小化块访问的能力,几乎没有其他树型结构的随机读取性能比B$^+$树好,但我们仍然可以通过整理磁盘写入操作的思路提高其写入性能。

\begin{figure}[htbp]
    \centering
    \begin{overpic}[scale=1]{lsmtree.png}
        \put(9,18){\scriptsize C$_0$树}
        \put(39,5){\scriptsize C$_1$树}
        \put(74,5){\scriptsize C$_2$树}
        \put(7,5){\scriptsize 顺序日志}
        \put(25,22){\scriptsize 刷新}
        \put(53,22){\scriptsize 刷新}
        \put(9.5,27){\scriptsize 内存}
        \put(38,-2.5){\scriptsize 容量呈指数增长}
    \end{overpic}
    \\[1.5em]
	\caption{B$^+$树及前缀B$^+$树示例\label{fig:lsmtree}}
\end{figure}

LSM树是一种超越B$^+$树的随机写入性能的索引结构。如图\ref{fig:lsmtree}所示,一些B$^+$树或类B$^+$树结构组织一个单独的LSM树。在LSM树上层有一个存储于内存的树结构,被称为C$_0$树。所有更新记录追加至一个顺序日志中,C$_0$树索引着日志中的每一个项以便高效访问。一旦C$_0$树的尺寸超过了某个阈值,一段连续的日志记录将被合并进C$_1$树。当C$_1$树的尺寸超过某个阈值,C$_1$树中一段连续的记录会以同样的方式合并进C$_2$树。合并操作总是在C$_i$树和C$_{i+1}$树之间发生,通常树的容量随$i$指数增长。

因为追加日志和合并操作都是对磁盘的顺序写入,因此LSM树的写入性能要比传统B$^+$树好得多。然而要访问一个键值对,我们只能从C$_0$开始对所有的树进行级联查找,直到发现目标记录,因此LSM的读取开销较B$^+$树大。为了避免不必要的遍历操作,LSM树一般使用布隆过滤器减弱读放大效应。

请注意第1章中提出的问题在LSM树中同样存在。如果键长度变长,较小的扇出会降低每棵树的容量,树间的合并操作将更加频繁。因为合并会触发大量I/O操作,导致性能的急剧下降。

\subsection{Couchstore}

\begin{figure}[htbp]
    \centering
    ~\\
    \begin{overpic}[scale=1]{couchstore.png}
        \put(55,2){\tiny 字节偏移量}
        \put(75,49){\scriptsize 最新数据}
        \put(75,42){\scriptsize 陈旧数据}
        \put(75,34){\scriptsize B$^+$树节点}
        \put(75,28){\scriptsize 文档}
        \put(66,11){\scriptsize 只追加}
        \put(4,19){\scriptsize DB文件的非结构化视图}
        \put(4,55){\scriptsize B$^+$树节点和文档的逻辑视图}
        \put(62,61){\scriptsize 单一Couchstore实例}
        \put(41,95){\scriptsize 键空间}
    \end{overpic}
	\caption{Couchstore概览\label{fig:couchstore}}
\end{figure}

Couchstore 是Couchbase中的单节点存储引擎,其整体架构继承自Apache CouchDB的存储模块。键空间被等分为若干用户定义的键范围数,我们称其为vBucket(或分区),每个vBucket拥有它自己的DB文件。每个DB文件所存储的键值对属于其对应的vBucket,每个键是一个任意长的字符串,每个值是JSON文档。为了确定文件中某个文档的位置,每个vBucket中有一个B$^+$树用于存储键和其对应文档的字节偏移量所构成的元组。因此,每一个DB文件包含文档和B$^+$树节点两部分,他们互相交错存储于文件中。

如图\ref{fig:couchstore}所示,在Couchstore中所有的更新操作都追加在DB文件的后面。$A$、$B$和$C$代表B$^+$树节点,在胶囊图形中的$D1$、$D2$和$D3$代表文档。如果$D1$文档更新了,新文档$D1'$将被写入文件末尾,而不会擦除或修改原文档$D1$。因为文档位置更新了,$B$节点必须更新为$B'$节点,同样追加至文件末尾。更新操作一直传递到根节点$A$,最终新根节点$A'$被写入到$B'$节点后。

相比于就地更新策略,只追加的B$^+$树可以实现非常高的写入吞吐量,因为所有的磁盘写入操作都是顺序的。此外,我们不需要牺牲读取性能,因为读取过程与传统B$^+$树是相同的。但是DB文件的占用空间随更新操作而增长,因此我们必须定期回收陈旧数据所占用的空间。当陈旧数据与总数据尺寸比超过一定预设阈值时,Couchstore触发器将执行这个压缩过程。当前目标DB文件中所有的文档将被移动到新DB文件,在压缩操作完成后,旧DB文件将被删除。在压缩过程中,所有对目标文件的写入操作将被阻塞,而读取操作是被允许的。每次仅对一个DB文件进行压缩。

与传统B$^+$树处理字符串的方式相同,为了维持容量不变,如果键长度增加,树的高度将增加。因为追加数据量与树高度成正比,因此这种只追加设计的性能衰减可能会更糟糕。由此导致压缩操作被更频繁的触发,整体性能陷入恶性循环。我们需要为变长字符串键设计一个更紧凑更有效的索引结构。
\section{ForestDB的设计}

为了高效的索引变长键,我们提出了ForestDB,它是为分布式NoSQL数据库设计的单节点后端键值存储引擎。ForestDB将作为Couchstore的替代品,因此两者的上层架构是类似的。两者方案的主要区别是:(1)ForestDB使用一种名为HB$^+$字典树新型混合索引结构,相比于传统B$^+$树,它可以高效的索引变长字符串。(2)通过结构化日志写入缓冲进一步提高ForestDB的写入吞吐量。结构化日志写入缓冲的基本概念类似LSM树中的C$_0$树和顺序日志,但后续不需要从写入缓冲中合并数据进主DB部分。

\begin{figure}[htbp]
    \centering
    ~\\
    \begin{overpic}[scale=1]{forestdb.png}
        \put(3.5,41){\begin{turn}{-47} \tiny 更新\end{turn}}
        \put(11.5,23){\scriptsize WB索引}
        \put(62,23){\scriptsize HB$^+$字典树}
        \put(38,27.5){\tiny 刷写}
        \put(7.5,7.5){\scriptsize 文档}
        \put(23,7.5){\scriptsize 文档}
        \put(52,7.5){\scriptsize 文档}
        \put(68.5,7.5){\scriptsize 文档}
        \put(85,7.5){\scriptsize 文档}
        \put(18,38){\scriptsize 位于内存}
        \put(62,38){\scriptsize 位于磁盘(只追加)}
        \put(64,50){\scriptsize 单一ForestDB实例}
        \put(41,83){\scriptsize 键空间}
    \end{overpic}
	\caption{Couchstore概览\label{fig:forestdb}}
\end{figure}

图\ref{fig:forestdb}展示了ForestDB的总体架构。与Couchstore类似,这是一个vBucket的单一实例。每一个ForestDB实例包含一个基于内存的写入缓冲索引(WB索引)和一个HB$^+$字典树。所有文档更新追加至DB文件末尾,位于内存的写入缓冲索引持续同步文档在磁盘中的位置。当写入缓存索引中项数量超过一定阈值后,这些项将被刷写到HB$^+$字典树并持久化到DB文件中。在3.3章中将详细描述具体的运作过程。ForestDB的压缩过程与Couchstore相同。当陈旧数据尺寸超过一定阈值,压缩过程将被触发,当前所有的文档将被移动到新的DB文件中。

\subsection{HB$^+$字典树}

ForestDB的主要索引结构——HB$^+$字典树,是一个使用B$^+$树节点的Patricia字典树变种。HB$^+$字典树的基本思想源于我们先前的工作,主要是为只追加存储而设计。所有B$^+$树的叶节点存储其他B$^+$树(子树)的根节点或文档。B$^+$树节点和文档都会以只追加的方式写入DB文件,它们交错存在与文件中以便在Couchstore实现MVCC。在HB$^+$字典树上层有一个B$^+$树,子树则作为Patricia字典树的新节点按需创建。图\ref{fig:the_hierarchical_organization_of_hbptrie}a展示了HB$^+$字典树的逻辑布局,图\ref{fig:the_hierarchical_organization_of_hbptrie}b说明字典树节点如何以MVCC模式存储于硬盘中。

\begin{figure}[htbp]
    \centering
    \begin{overpic}[scale=1]{the_hierarchical_organization_of_hbptrie.png}
        \put(2,21.3){\tiny 键:}
        \put(1.2,15){\tiny 块1}
        \put(5.6,15){\tiny 块2}
        \put(10.3,15){\tiny 块3}
        \put(2,11.5){\tiny 固定尺寸(本例中4字节)}
        \put(5,8.5){\tiny B$^+$树(HB$^+$字典树的节点)}
        \put(5,5.2){\tiny 文档}
        \put(26,21.3){\tiny 根B$^+$树}
        \put(34.5,19.5){\tiny 块1}
        \put(26.5,14.5){\tiny 块2}
        \put(43,14.5){\tiny 块2}
        \put(30,9){\tiny 块3}
        \put(37.5,4.6){\tiny 文档}
        \put(70,27){\tiny 根B$^+$树}
        \put(54.5,7){\tiny DB文件的非结构化视图}
        \put(94,25){\tiny 最新数据}
        \put(94,22.2){\tiny 陈旧数据}
        \put(94,18){\tiny B$^+$树(HB$^+$字典树的节点)}
        \put(94,15.2){\tiny B$^+$树的节点}
        \put(94,12.4){\tiny 文档}
        \put(101,3.8){\tiny 只追加}
        \put(94,-0.4){\tiny 字节偏移}
        \put(16,-3){\scriptsize (a)逻辑设计}
        \put(70,-3){\scriptsize (b)磁盘数据模型}
    \end{overpic}
    \makebox[2em]{}
    \\[2em]
	\caption{HB$^+$字典树的分层结构\label{fig:the_hierarchical_organization_of_hbptrie}}
\end{figure}

HB$^+$字典树将输入的键拆分成固定大小的若干个块。块尺寸是可配置的(例如4或8字节),一组连续的块分别作为B$^+$树邻接层的键。检索文档时,从根B$^+$树中寻找键为第一个(最左边的)块的节点,从而获得其对应的字节偏移,递归向下层进行,直到定位到文档的位置。此外,如果字节偏移指向的是其他子树的根节点,我们使用下一个块在子树中递归搜索,直到目标文档被找到。

因为B$^+$树的键尺寸固定为小于键字符串的块尺寸,B$^+$树节点的扇出比传统B$^+$树要大,所以我们可以缩短树的高度。此外,与传统Patricia字典树相同,通过公共分支共享公共前缀的方式跳过和压缩了键。当且仅当至少两个分支经过树时会创建子树。所有的文档使用可区分其他文档的块的最小集来索引。

\begin{figure}[htbp]
    \centering
    \begin{overpic}[scale=1]{hbptrie_insertion_examples.png}
        \put(10,9){\scriptsize B$^+$树}
        \put(10,2){\scriptsize 文档}
        \put(50,31){\scriptsize 跳过的前缀}
    \end{overpic}
	\caption{HB$^+$字典树的插入示例:(a)初始状态;(b)插入$aaab$;(c)插入$aabb$\label{fig:hbptrie_insertion_examples}}
\end{figure}

图\ref{fig:hbptrie_insertion_examples}展示了一个插入过程的例子。假设块尺寸只有一个字节(本例中,一个字符),B$^+$树作为HB$^+$字典树的节点,用三角形表示。B$^+$树中的文本记录了如下内容:(1)当前B$^+$树使用第几个块作为键;(2)当前树的公共前缀。图\ref{fig:hbptrie_insertion_examples}a表示索引只存储键$aaaa$时的初始状态,根B$^+$树直接使用第一个块索引文档,因为键$aaaa$是唯一以$a$块起始的键。我们可以确保首个块$a$只对应键$aaaa$。因而我们可以避免后续的树遍历操作。

我们将一个新键$aaab$插入到HB$^+$字典树,因为键$aaab$也起始与块$a$(参见图\ref{fig:hbptrie_insertion_examples}b),这时会创建一个新的子树。因为$aaaa$和$aaab$的公共最长前缀是$aaa$,新的子树将使用第四个块作为键,并存储该树距父树(本例中为根树)跳过的前缀$aa$。图\ref{fig:hbptrie_insertion_examples}c展示了插入键$aabb$时的情况。虽然键$aabb$也起始于块$a$,但是它不匹配此前跳过的前缀$aa$。于是我们在根和此前存在的子树之间创建一个新的子树。该子树使用第三个块作为键,因为$aabb$与此前存在的公共前缀$aaa$的最长公共前缀是$aa$。被跳过的前缀$a$储存于新的树,此前存在于在第四个块树的前缀被擦除,因为此时两颗树之间没有跳过的前缀。

\begin{figure}[htbp]
    \centering
    \begin{overpic}[scale=1]{an_example_of_random_key_indexing_using_hbptrie.png}
        \put(6,31.5){\tiny 键}
        \put(19,31.5){\tiny 首块}
        \put(3,0){\tiny (块尺寸:2字节)}
        \put(59,1){\scriptsize B$^+$树}
        \put(80,1){\scriptsize 文档}
    \end{overpic}
	\caption{使用HB$^+$字典树的随机键索引示例\label{fig:an_example_of_random_key_indexing_using_hbptrie}}
\end{figure}

当公共前缀足够长时,HB$^+$字典树带来的好处是显而易见的。即便键的分布随机且无公共前缀,HB$^+$字典树同样能带来很多好处。图\ref{fig:an_example_of_random_key_indexing_using_hbptrie}展示了一个块尺寸为2字节的随机键的例子。因为其中的键没有公共前缀,第一个块就可以区分他们。在本例中,HB$^+$字典树只包含一个B$^+$树,不需要为了比较而创建任何子树。假设块尺寸是$n$位,且键分布均匀随机,那么在B$^+$树中仅通过首块就可以检索到至多$2n$个键。相比于传统B$^+$树,HB$^+$字典树惊人的将索引结构的空间占用降低了一个数量级。

总的来说,当将足够长且含有公共前缀或键随机分布以至于没有公共前缀时,HB$^+$字典树都可以有效的降低由不必要的树遍历引起的磁盘访问开销。

\subsection{HB$^+$字典树中树倾斜情况的优化}

\subsubsection{概览}

因为字典树并不是一个平衡结构,HB$^+$字典树在特定情况下会同传统字典树一样产生倾斜。图\ref{fig:skewed_hbptrie_examples}展示了块尺寸为1字节时,导致树倾斜的两个典型例子。如果我们连续插入一组块不断重复的键,例如$b$、$bb$和$bbb$,HB$^+$字典树就会产生类似图\ref{fig:skewed_hbptrie_examples}a展示的树倾斜情况。从而导致沿倾斜分支的查询操作引起的磁盘访问量增加。

\begin{figure}[htbp]
    \centering
    \begin{overpic}[scale=1]{skewed_hbptrie_examples.png}
        \put(11,-5){\scriptsize (a)}
        \put(66,-5){\scriptsize (b)}
    \end{overpic}
    \\[2em]
	\caption{倾斜的HB$^+$字典树示例\label{fig:skewed_hbptrie_examples}}
\end{figure}

图\ref{fig:skewed_hbptrie_examples}b展示了倾斜HB$^+$字典树的另一个例子。如果键仅由0和1字符组成,那么每一个块只会有两个分支,因此所有的B$^+$树将只包含两个键值对。由于B$^+$树的扇出远高于2,则创建了大量的接近于空的块,导致块利用率将大大降低。于是,相较于传统B$^+$树,空间开销和块访问数都将大幅度增加。

事实上,我们可以通过增大块尺寸来降低HB$^+$字典树倾斜的概率。在图\ref{fig:skewed_hbptrie_examples}中,如果我们将块尺寸设为8字节,每个块将有$2^8=256$个分支,B$^+$树会得到更合理的利用。然而随着块尺寸的增加,节点扇出将减少,进而导致总体性能的下降。

\begin{figure}[htbp]
    \centering
    \begin{overpic}[scale=1]{examples_of_optimization_for_avoiding_skew.png}
        \put(9,5.5){\scriptsize 非叶B$^+$树(键尺寸固定)}
        \put(9,1){\scriptsize 叶B$^+$树(键尺寸不定)}
        \put(76,5.5){\scriptsize 非叶B$^+$树(键尺寸固定)}
        \put(76,1){\scriptsize 叶B$^+$树(键尺寸不定)}
        \put(19,-4){\scriptsize (a)初始状态}
        \put(66,-4){\scriptsize (b)左侧叶B$^+$扩展后的状态}
    \end{overpic}
    \\[2em]
	\caption{避免倾斜的优化示例\label{fig:examples_of_optimization_for_avoiding_skew}}
\end{figure}

为了解决这个问题,我们提出了一个优化方案。首先我们定义,叶B$^+$树即没有子树的非根B$^+$树。不同于非叶B$^+$树存储固定尺寸的块,叶B$^+$树存储变长字符串,这个字符串即未作为其父B$^+$树中块的键后缀。图\ref{fig:examples_of_optimization_for_avoiding_skew}展示了叶B$^+$树的组织形式。途中白色三角形和黑色三角形分别表明非叶B$^+$树和叶B$^+$树。非叶B$^+$树包括根B$^+$树索引或使用相应块做键的子树,而叶B$^+$树使用剩余的子串作为键。例如,图\ref{fig:examples_of_optimization_for_avoiding_skew}a中左侧的叶B$^+$树分别使用文档的后缀$aaa$、$aabc$、$abb$(从第二个块开始)索引文档$aaaa$、$aaabc$和$aabb$。在这种方式下,几遍我们插入可以引起树倾斜的键集,也不会创建更多子树,从而避免多余的树遍历。

\subsubsection{叶B$^+$树的延伸}

但是这种数据结构几乎继承了传统B$^+$树的所有缺点。为了避免这些问题,当叶B$^+$树中容纳的键总数超过一定阈值时,我们将叶B$^+$向下延伸。扩展步骤如下:我们首先找到目标叶B$^+$树中键的最长公共前缀。为第一个不同的块创建一个新非叶B$^+$树,相应的文档使用新的块重建索引。如果有多余一个键使用相同的块,我们则使用剩余的子串作为键创建一个新叶B$^+$树。

图\ref{fig:examples_of_optimization_for_avoiding_skew}b展示了一个将图\ref{fig:examples_of_optimization_for_avoiding_skew}a左侧叶B$^+$节点延伸的例子。因为$aaaa$、$aaabc$和$aabb$的公共前缀是$aa$,索引使用第三个块创建一个新的非叶B$^+$树,文档$aabb$通过其第三个块$b$索引。但$aaaa$和$aaabc$拥有公共的第三个块$a$,因此我们创建一个新的叶B$^+$树,使用余下的子串$a$和$bc$索引这些文档。

这种方案将键空间分为两种不同的区域:倾斜区域和普通区域。倾斜区域即叶B$^+$树索引的键集,剩下的键集合则为普通区域。因为原生HB$^+$字典树在先前叙述的键倾斜情况下非常低效,我们必须非常谨慎的设置延伸阈值以防止倾斜键蔓延到普通区域。

\subsubsection{对延伸阈值的分析}

最佳的延伸阈值,应该是给定键集字典树索引和类树索引开销的平衡点。为了找出最佳的延伸阈值,我们对两种索引的开销进行了数学分析。从直觉来看,字典树结构的高度和空间占用(例如,节点数)与每一个块中的独立分支数高度相关,而对于类树结构,最关键的影响因素只有键长度。于是,我们通过键集中键的长度和每个块的分支数分析使得字典树的高度和空间占用都小于类树结构的平衡点。

\begin{figure}[htbp]
    \centering
    {
    \bfseries
    表1 \\
    符号定义 \\[1.5em]
    }
    \begin{tabular}{|p{3em}p{14em}|p{3em}p{14em}|}
    \hline
    符号 & 定义 & 符号 & 定义 \\
    \hline
    $n$ & 文档数 & $f_L^{new}$ & 延伸后新叶B$^+$树节点的扇出 \\
    $B$ & 块尺寸 & $s$ & 叶B$^+$树的空间开销 \\
    $k$ & 叶B$^+$树中键长度 & $s_{new}$ & 延伸后新结合的数据结构的空间开销 \\
    $c$ & HB$^+$字典树的块的长度 & $h$ & 叶B$^+$树的高度 \\
    $v$ & 值(例如,字节偏移量)或指针的尺寸 & $h_{new}$ & 延伸后新结合的数据结构的高度 \\
    $f_N$ & 非叶B$^+$树节点的扇出 & $b$ & 给定键集下每个块的独立分支数量 \\
    $f_L$ & 叶B$^+$树节点的扇出 & & \\
    \hline
    \end{tabular}
\end{figure}

表1罗列了我们分析中使用的符号。假设一个叶B$^+$树索引了$n$个文档,每个B$^+$树节点恰好与存储设备块对齐,其尺寸是$B$。所有的键长度相同,为$k$,因此长度$k$大于等于$c\lceil\log_bn\rceil$,其中$c$和$b$分别表示HB$^+$字典树的块尺寸和每个块的独立分支数量。由此易得每个叶B$^+$树节点扇出$f_L$,如下:

\begin{equation}
f_L=\left\lfloor\frac{B}{k+v}\right\rfloor, \label{equation1}
\end{equation}

其中$v$表示字节偏移量或指针的尺寸。对于给定$n$个文档,我们可以得出叶B$^+$树\footnote{为了简化计算,我们忽略了叶B$^+$树中索引节点所占用的空间$s$,因为与叶节点相比,它们占用的空间微乎其微。}整体的空间占用,以及叶B$^+$树的高度$h$,如下:

\begin{equation}
s\simeq\left\lceil\frac{n}{f_L}\right\rceil B, \label{equation2}
\end{equation}
\begin{equation}
h=\left\lceil\log_{f_L}n\right\rceil. \label{equation3}
\end{equation}

延伸过程中,当每个块含有$b$个分支时,将创建$b$个新的叶B$^+$树。同时也将创建一个非叶B$^+$树\footnote{这里假设最差的情况,即非叶B$^+$树没有直接指向文档。}指向$b$个新的叶B$^+$树。我们先前提到,新的叶B$^+$树以其父非叶B$^+$树\footnote{假设这里没有跳过公共前缀。}块使用的右侧剩余子串作为键,因此新的叶B$^+$树扇出$f_L^{new}$可以使用下面的公式表示:

\begin{equation}
f_L^{new}=\left\lfloor\frac{B}{(k-c)+v}\right\rfloor. \label{equation4}
\end{equation}

因为非叶B$^+$树使用一个块作为键,因此其扇出$f_N$可以使用下面的公式表示:

\begin{equation}
f_N=\left\lfloor\frac{B}{c+v}\right\rfloor. \label{equation5}
\end{equation}

通过$f_N$和$f_L^{new}$,可知新混合数据结构的空间占用和高度,分别记作$s_{new}$和$h_{new}$,如下:

\begin{equation}
s_{new}\simeq\left(\left\lceil\frac{b}{f_N}\right\rceil+b\left\lceil\frac{n}{b\cdot f_L^{new}}\right\rceil\right) B, \label{equation6}
\end{equation}
\begin{equation}
h_{new}=\left\lceil\log_{f_N}b\right\rceil+\left\lceil\log_{f_N^{new}}\frac{n}{b}\right\rceil. \label{equation7}
\end{equation}

图\ref{fig:space_overhead_and_height}a是当$b$取2、8、64和256时$s_{new}$对$s$归一化的值。我们如下设置参数值:$B=4096$、$k=64$、$c=8$、$v=8$,则可算得$f_L=56$、$f_N=256$、$f_L^{new}=64$。空间开销随$b$的增加而增长。这是因为延伸过程创建了$b$个新的叶B$^+$树。而一个B$^+$树至少占一个块,$b$个新的叶B$^+$树就至少占$b$个块,又有$\left\lceil\frac{n}{f_L}\right\rceil$个块被原有的叶B$^+$树占用。当$b<\frac{n}{f_L}\Leftrightarrow n>b\cdot f_L$时,$s_{new}$小于$s$,这个点是新的叶B$^+$树对块占用小于延伸前块占用的分界点。

\begin{figure}[htbp]
    \centering
    \begin{overpic}[scale=0.6]{space_overhead_and_height.png}
        \put(-3,20){\scriptsize \parbox[l]{1em}{归一化空间开销}}
        \put(52,20){\scriptsize \parbox[l]{1em}{归一化平均高度}}
        \put(18,-3){\scriptsize 文档数($n$)}
        \put(75,-3){\scriptsize 分支数($b$)}
        \put(7,-10){\scriptsize \parbox[l]{15em}{(a)$b$取不同值时,$s_{new}/s$值随文档数$n$的变化}}
        \put(62,-10){\scriptsize \parbox[l]{15em}{(b)当文档数$n$在$b\cdot f_L$、$b\cdot f_L^2$之间时,$h_{new}/h$平均值随$b$的变化}}
    \end{overpic}
    \\[4.5em]
	\caption{叶B$^+$树扩展前后,其空间占用(a)及高度(b)的归一化值的折线图。\label{fig:space_overhead_and_height}}
\end{figure}

然而,尽管延伸后空间开销减少了,整体的高度却是增加的。图\ref{fig:space_overhead_and_height}b展示了延伸后不同$b$值下混合数据结构的平均高度$h_{new}$。高度对延伸前叶B$^+$树的高度$h$进行了归一化处理。当$b$大于原有叶B$^+$树扇出$f_L$时,新的高度$h_{new}$将小于先前的高度$h$,因为延伸后非叶B$^+$树中的分支数大于扩展前叶B$^+$树扇出。

在权衡上述因素后,我们认为在以下条件下应该延伸叶B$^+$树:(1)$n>b\cdot f_L$(2)$b\ge f_L$.值得注意的是,我们必须扫描所有在叶B$^+$树中的键才能获得$k$和$b$的准确值,这将引起大量的磁盘I/O操作。为了避免这种开销,我们只扫描叶B$^+$树的根节点。根节点包含的键集中,键的间隔是近似相等的,因此我们可以估计近似值。而根节点已经因先前的树操作被缓存,因此我们不需要额外的磁盘I/O操作。

\subsection{结构化日志写入缓冲}

虽然HB$^+$字典树可以降低树高度以及整体空间占用,但写入操作仍然会引起不少于一个B$^+$树节点追加至DB文件。为了降低写入操作开销,ForestDB使用结构化日志写入缓冲减少每次写入操作引起的追加数据量。它和LSM树中的$C_0$树和顺序日志非常类似,但是被插入到写缓冲部分的文档不需要被合并到主DB部分,因为主DB本身也基于结构化日志设计。

\begin{figure}[htbp]
    \centering
    \begin{overpic}[scale=0.6]{write_buffer_examples.png}
            \put(4,6.5){\scriptsize 磁盘(只追加)}
            \put(49,6.5){\scriptsize 磁盘(只追加)}
            \put(15,10){\scriptsize 位于内存}
            \put(60,10){\scriptsize 位于内存}
            \put(30,14){\tiny \parbox[t]{3em}{写缓存\\\makebox[3em][c]{索引}}}
            \put(75,14){\tiny \parbox[t]{3em}{写缓存\\\makebox[3em][c]{索引}}}
            \put(26.5,-1){\scriptsize 写缓冲中的文档}
            \put(4,3.5){\scriptsize 文档}
            \put(26,3.5){\scriptsize 文档}
            \put(35,3.5){\scriptsize 文档}
            \put(49,3.5){\scriptsize 文档}
            \put(71.5,3.5){\scriptsize 文档}
            \put(80,3.5){\scriptsize 文档}
            \put(13,3.5){\scriptsize 索引节点}
            \put(58.5,3.5){\scriptsize 索引节点}
            \put(88.5,3.5){\scriptsize 索引节点}
            \put(13,-4){\scriptsize (a)刷新写缓冲前}
            \put(69,-4){\scriptsize (b)刷新写缓冲后}
    \end{overpic}
    \\[3em]
	\caption{写入缓冲示例。\label{fig:write_buffer_examples}}
\end{figure}

图\ref{fig:write_buffer_examples}a 展示了一个例子。白色方块中文档表示一组用于文档的临近的磁盘块,灰色方块中索引节点表示用于由B$^+$树节点组成的HB$^+$字典树的块。进行提交操作时,一个包含DB头信息的块被追加至文件末尾,我们用深灰色方块中的H表示它。

所有文档更新被简单的追加到文件的末尾,而对于HB$^+$字典树的更新是滞后的。我们使用一种位于内存的写入缓冲索引记录那些已经被写入文件但尚未被HB$^+$字典树索引的文档位置。当响应查询文档的请求时,ForestDB首先在写入缓冲索引中查找,如果没有命中则继续查找HB$^+$字典树。

当且仅当提交日志的累积尺寸超过了某个预设阈值(例如,1024个文档)时,下一次提交操作将触发对写缓冲中各项的刷新,以及原子性的映射到HB$^+$树中。在刷新写缓冲后,写缓冲中更新的文档对应的索引节点将被追加至文件末尾,如图\ref{fig:write_buffer_examples}b所示。据前述,文档本身并不需要移动或者合并,只需要使用更新的索引节点链接,因为ForestDB已经使用了结构化日志的设计。这大幅度减少了刷新写缓冲的整体开销。

如果在刷新之前写缓冲崩溃了,我们可以从文件末尾反向检索每一个块,直到找到最后一个在索引节点后写入的有效的DB头。只要DB头被找到,ForestDB就可以通过头后面被写入的文档重建写缓冲索引项。我们还为每一个文档维护一个CRC32校验值,因此只要文档被正常写入,就是可以被恢复的。

通过结构化日志写入缓冲,大量的索引更新操作被转化为批量的,因此可以降低每文档更新引起的磁盘I/O数。这使得ForestDB的写入性能可以比肩甚至超越LSM树的写性能。

\section{评测}

在本章中,我们通过评测结果来展示ForestDB的关键特性。ForestDB实现为Couchbase服务器的单机键值存储库。评测运行在64位Linux 3.8.0-29平台上,配备Intel Core i7-3770 @ 3.40 GHz CPU(4 cores, 8 threads),32GB RAM,Western Digital Caviar Blue 1TB HDD\footnote{WD10EALX,最大数据传输速率:126 MB/s。},磁盘被格式化为Ext4文件系统。所有的评测中,HB$^+$字典树的块尺寸都被设为8字节,HB$^+$字典树中的叶B$^+$树的实现为前缀B$^+$树。

\subsection{索引结构比较}

首先我们对比HB$^+$字典树和传统B$^+$树、前缀B$^+$树的整体空间开销和遍历路径平均长度。我们人为的插入一百万个文档,以及其对应的一百万个键被插入到索引中,每个键映射一个8字节的文档位置偏移量,以此作为初始状态。为了验证避免树倾斜的优化方案,我们使用表2所描述的四种键模式:随机、最糟、小健、两层。每一个B$^+$树节点对齐到4KB块尺寸,所有包含HB$^+$字典树的索引使用就地更新的方式写入以精确估计实时的索引尺寸。

\begin{figure}[htbp]
    \centering
    {
    \bfseries
    表2 \\
    键模式特征 \\[1.5em]
    }
    \begin{tabular}{|p{2em}p{30em}p{6em}|}
    \hline
    名称 & 描述 & 键长度 \\
    \hline
    随机 & 该键模式中所有键随机产生,键之间没有公共前缀,对于HB$^+$树而言,这是最佳的键模式。 & 8-256字节 \\
    最糟 & 该键模式中所有键含有20层嵌套前缀,每一层前缀仅有两个分支。每一层的平均前缀尺寸时10字节。对于HB$^+$树而言,这是最糟的键模式。 & 平均198字节 \\
    小键 & 该键模式中有100个随机产生的前缀,每10000个键共享一个公共前缀。平均前缀尺寸是10字节,非前缀部分键字符串是随机生成的。 & 平均65字节 \\
    两层 & 该键模式中所有的键含有两层嵌套前缀,每层包含192个分支。每层的平均前缀尺寸为10字节,非前缀部分键字符串是随机生成的。 & 平均64字节 \\
    \hline
    \end{tabular}
\end{figure}

图\ref{fig:comparison_random_key_pattern}a展示了每个使用随机键模式初始化后的索引结构的磁盘空间占用情况,其键长度分布在8-256字节之间。B$^+$树和前缀B$^+$树的空间开销随键长度线性增长,而HB$^+$字典树保持不变。这是因为所有的键可以只通过它的第一个块索引,因此键长度不影响HB$^+$字典树的空间开销。

\begin{figure}[htbp]
    \centering
    \begin{overpic}[scale=0.6]{comparison_random_key_pattern.png}
        \put(19,-3){\scriptsize 键长度(字节)}
        \put(73,-3){\scriptsize 键长度(字节)}
        \put(17,-6){\scriptsize (a)空间开销}
        \put(70,-6){\scriptsize (b)HDD磁盘时延}
        \put(-5,20){\scriptsize \parbox[l]{1em}{空间开销(MB)}}
        \put(50,20){\scriptsize \parbox[l]{1em}{磁盘时延(ms)}}
    \end{overpic}
    \\[3em]
	\caption{在随机键模式下,B$^+$树、前缀B$^+$树和HB$^+$字典树的对比实验。\label{fig:comparison_random_key_pattern}}
\end{figure}

为了评估从根节点到叶的平均遍历长度,我们随机读取键值对,记录树操作期间的磁盘访问时间开销。因为当磁盘I/O操作请求命中操作系统页缓存时,磁盘操作将被跳过,所以我们开启了O\_DIRECT标记使得所有的树节点读操作引起实际的磁盘访问。图\ref{fig:comparison_random_key_pattern}b展示了该结果。即便键长度增大,HB$^+$字典树的磁盘时延基本稳定,而B$^+$树的磁盘时延因扇出降低导致的树高度增加而急剧增长。值得注意的是,当键长度小鱼256字节时,前缀B$^+$树也展现了一个基本稳定的磁盘时延。这是因为随机键基本不包含公共前缀,因此索引节点中的最小可区分子串变得非常小,所以每个索引节点的扇出依旧很大。虽然叶节点的低扇出放大了空间开销,索引节点的高扇出还是可以缩短树的高度。

\begin{figure}[htbp]
    \centering
    \begin{overpic}[scale=0.6]{comparison_using_worst_small_2level_key_patterns.png}
        \put(13,-1){\scriptsize 最糟}
        \put(23,-1){\scriptsize 小键}
        \put(33,-1){\scriptsize 两层}
        \put(68,-1){\scriptsize 最糟}
        \put(78,-1){\scriptsize 小键}
        \put(88,-1){\scriptsize 两层}
        \put(17,-6){\scriptsize (a)空间开销}
        \put(70,-6){\scriptsize (b)HDD磁盘时延}
        \put(-5,20){\scriptsize \parbox[l]{1em}{空间开销(MB)}}
        \put(50,20){\scriptsize \parbox[l]{1em}{磁盘时延(ms)}}
    \end{overpic}
    \\[3em]
	\caption{在随机键模式下,B$^+$树、前缀B$^+$树和HB$^+$字典树的对比实验。\label{fig:comparison_using_worst_small_2level_key_patterns}}
\end{figure}

下面我们使用相同的评估手段测试最糟、小键和两层键模式。图\ref{fig:comparison_using_worst_small_2level_key_patterns}展示了每种索引结构的空间开销和磁盘时延。这里HB$^+$字典树w/o模式表示没有避免倾斜优化的HB$^+$字典树,HB$^+$字典树w/模式代表开启优化的HB$^+$字典树。

在最糟模式下,相比于B$^+$树,前缀B$^+$树的空间开销有大幅度的降低,这是因为键中有大量的公共前缀,并且其长度足够长。在对比中,未优化的HB$^+$字典树比B$^+$树占用了更多的空间,这是因为这种键模式导致字典是倾斜,整体块使用率变得非常低。更糟的是,未优化的HB$^+$字典树的磁盘时延是B$^+$树的2.8倍,是前缀B$^+$的3.4倍。然而,经过优化的HB$^+$字典树通过使用叶B$^+$树有效的降低开销。无论是空间开销或是磁盘时延HB$^+$字典树都小于前缀B$^+$树,因为HB$^+$字典树中每个叶B$^+$树是采用前缀B$^+$树实现的。

不同于最糟模式,小键模式中每个块存在大量的分支,因此字典树状索引优于树状索引。如图\ref{fig:comparison_using_worst_small_2level_key_patterns}所示,未优化HB$^+$字典树的空间开销与磁盘时延已经优于B$^+$树和前缀B$^+$树。而经过优化的HB$^+$字典树同样展示了类似的性能,这是我们对因为叶B$^+$树的延伸时机把握的特别好,从而使得该数据结构的空间开销几乎与原生HB$^+$树相同。

因为在两层模式下,每个块同样有很多分支,所以未优化HB$^+$字典树的磁盘时延优于B$^+$树和前缀B$^+$树。然而,每第三个块平均仅有$1,000,000/192^2\simeq27$个分支,这远远小于非叶B$^+$树节点的最大扇出(本测试中为256)。因此存在很多几乎为空的块,从而导致原生HB$^+$字典树的空间开销大于B$^+$树和前缀B$^+$树。正如我们在3.2.3章中提到的,这种优化方式对这种情况处理的很好,避免叶B$^+$树被扩展。所以优化后HB$^+$字典树的空间开销小于前缀B$^+$树。

\subsection{整体系统性能表现}

下面我们对ForestDB整体系统性能进行评测,我们将之与Couchstore 2.2.0、LevelDB 1.18和RocksDB 3.5对比。为了公平起见,LevelDB、RocksDB和ForestDB的API层均被转化为Couchstore操作。在LevelDB和RocksDB中开启了Snappy文档压缩,但是文档是随机产生的,因此只有键部分可以被压缩。对于Couchstore和ForestDB,当陈旧数据占项数据尺寸的30\%时,将触发压缩操作,而LevelDB和RocksDB则使用后台独立线程持续进行压缩操作。ForestDB的写缓冲阈值配置为默认值4096个文档,所有系统的性能评估中都关闭了O\_DIRECT标记。为了减少LevelDB和RocksDB的写放大问题,我们为其增加了每键位数为10的布隆过滤器。除不支持缓存功能的Couchstore外,我们将每个系统的自定义块缓存设为8GB。

每个系统使用2亿个键及其对应文档初始化,文档的尺寸时1024字节(不含键长)。总体工作集的尺寸在190到200GB之间,几乎时RAM容量的6倍。所有测评都使用两层键模式。因为HDD太慢(低于100 ops每秒),难以初始化每个系统,得到有意义的结果,我们使用Samsung 850 Pro 512 GB SSD\footnote{MZ-7KE512B,最大读取速率:550 MB/秒,最大写入速率:520 MB/秒,随机读取:100,000 IOPS,随机写入:90,000 IOPS。}替代HDD进行整体系统评测。我们确保在SDD上进行的整体相对评测结果与HHD类似。

\subsubsection{键长度}

首先我们测试在不同键长度(16-1024字节)下,各系统的整体性能表现。为了清晰的观察键长度产生的影响,在这个测评中,我们随机产生键,使用尺寸为512字节的文档。随机执行大量的读取与更新操作,其中更新操作占总操作的20\%。批量更新操作以同步写入的方式应用,我们随机批量查询10到100个文档。而Couchstore在键长超过512字节时不能正确工作,因此在对Couchstore的测评中我们使用512字节的键代替1024字节的键。

图\ref{fig:various_key_lengths_performance_comparison}a通过每秒操作数体现整体吞吐量。随着键长增大,Couchstore、LevelDB和RocksDB下降了3-11.3倍,而得益于HB$^+$字典树,ForestDB仅下降了37\%。虽然HB$^+$字典树几乎不受键长影响,但ForestDB的吞吐量仍然随键长增加有略微下降,这是由于文档中包含键字符串作为元数据,而键的增长导致了文档的增长。

\begin{figure}[htbp]
    \centering
    \begin{overpic}[scale=0.6]{various_key_lengths_performance_comparison.png}
        \put(20,-3){\scriptsize 键长度(字节)}
        \put(76,-3){\scriptsize 键长度(字节)}
        \put(8,-6){\scriptsize (a)含20\%更新的无序操作的整体吞吐量}
        \put(63,-6){\scriptsize (b)每单独更新操作导致的平均写放大效应}
        \put(-3,20){\scriptsize \parbox[l]{1em}{每秒操作数}}
        \put(53,20){\scriptsize \parbox[l]{1em}{写放大效应}}
    \end{overpic}
    \\[3em]
	\caption{不同键长下的性能对比。\label{fig:various_key_lengths_performance_comparison}}
\end{figure}

因为所有的方案都采用异地更新的方法,因此空间开销会随着更新操作的增加而增长。图\ref{fig:various_key_lengths_performance_comparison}b展示了平均写放大现象以及压缩操作开销。随键长增加,Couchstore、LevelDB和RocksDB产生更多的磁盘写入操作,而ForestDB的写操作数几乎是稳定的。注意其中LevelDB的写操作数远大于其他方案,因为基于LSM树的LevelDB在更新过程中触发了大量的合并和压缩操作。虽然RocksDB同样基于LSM树结构,但它使用变种的结构以优化写放大问题。

\subsubsection{读/写比率}

为了探究在不同读写比率下的性能特点,我们使用0-100\%比例的更新操作进行整体吞吐量评测。键长度固定为32字节,在键空间内随机执行操作。我们同先前的评测一样,使用同步写操作。

图\ref{fig:various_update_ratios_concurrent_reader_threads_performance_comparison}a 呈现了测试结果。因为LevelDB、RocksDB和ForestDB使用结构化日志方案,它们的整体吞吐量随更新比例的增加而增长。所有更新操作被顺序的记录,无需检索或更新主索引,因此顺序写入磁盘的操作随更新比例的增加而增加。与此不同的是,无论更新比例如何,Couchstore的吞吐量几乎是稳定的,因为所有的更新都立即反应在B$^+$树上。虽然Couchstore同样采用只追加的设计,但更新或追加新的B$^+$树节点需要读取旧版本的节点。又因为旧节点随机的散落在磁盘上,随机读抵消了顺序更新和追加日志来带的性能提升。

值得注意的是,虽然ForestDB中的HB$^+$字典树采用分级混合B$^+$树结构,但其读性能要优于使用布隆过滤器的LevelDB和RocksDB。这是因为ForestDB的整体索引尺寸比较紧凑,HB$^+$字典树中每一级的B$^+$树仅存储相应的块,而非键字符串。所以,这同时增加了操作系统页缓存和DB自身缓存的命中率,进而降低了单操作引起的平均实际磁盘访问数。

当工作负载变为只读时,所有系统的整体性能都略微优于仅包含少部分更新操作的工作负载。如果没有更新操作,那么就不需要执行压缩或合并操作,我们便可以节省由压缩操作引起的额外的磁盘I/O。

\subsubsection{并发读/写操作}

ForestDB和Couchstore都基于MVCC模式,并发读写操作不会相互阻塞。为了探究多线程性能,我们创建了一个写线程,其以2000 ops每秒不断地进行随机写入操作,同时创建多个读线程以最大速率进行随机读操作。

\begin{figure}[htbp]
    \centering
    \begin{overpic}[scale=0.6]{various_update_ratios_concurrent_reader_threads_performance_comparison.png}
        \put(20,-3){\scriptsize 更新比率(\%)}
        \put(76,-3){\scriptsize 读线程数}
        \put(6,-6){\scriptsize (a)不同更新比率下无序操作的整体吞吐量}
        \put(68,-6){\scriptsize (b)并发读线程的吞吐量}
        \put(-2,20){\scriptsize \parbox[l]{1em}{每秒操作数}}
        \put(51,20){\scriptsize \parbox[l]{1em}{每秒操作数}}
    \end{overpic}
    \\[3em]
	\caption{不同更新比率(a)和并发线程数(b)下的性能对比。\label{fig:various_update_ratios_concurrent_reader_threads_performance_comparison}}
\end{figure}

图\ref{fig:various_update_ratios_concurrent_reader_threads_performance_comparison}b 展示了多个并发读取线程的读吞吐量和。当只使用单线程读取时,LevelDB和RocksDB的整体读取吞吐量大大小于图\ref{fig:various_update_ratios_concurrent_reader_threads_performance_comparison}a中数据。这是因为写操作和写操作触发的额外合并或压缩操作阻塞了并发读操作。相反,ForestDB和Couchstore甚至比单线程的成绩\footnote{图\ref{fig:various_update_ratios_concurrent_reader_threads_performance_comparison}b中去除了写线程2000 ops每秒的吞吐量,因此单读线程的整体吞吐量要高于图\ref{fig:various_update_ratios_concurrent_reader_threads_performance_comparison}a}更好,这是因为读线程几乎不被写线程影响。

随着并发读线程数的增加,所有系统的读吞吐量都有提升。这意所有系统中的味着并发读操作都不会阻塞自身。因为存在磁盘I/O瓶颈,部分点的读吞吐量是饱和的。

\subsubsection{范围扫描}

基于LSM树方案的主要优点之一就是范围扫描性能高,因为除$C_0$日志外的所有树形组件都采用有序的方式维护。相反,类似ForestDB和Couchstore这样的只追加方案并不擅长范围扫描,因为更新被简单的追加到DB文件的末尾,因此键空间内连续的文档在磁盘上可能并不处于相邻的物理位置。然而,通过压缩操作,文档会以有序的形式迁移到新的文件,因此范围扫面性能会较之前更好。

\begin{figure}[htbp]
    \centering
    \begin{overpic}[scale=0.6]{range_scan_performance.png}
        \put(36,-3){\scriptsize 100文档}
        \put(65,-3){\scriptsize 1000文档}
        \put(-4,30){\scriptsize \parbox[l]{1em}{每秒操作数}}
    \end{overpic}
    \\[2em]
	\caption{范围扫描性能。 \label{fig:range_scan_performance}}
\end{figure}

为了评测范围扫描性能,我们随机挑选一个文档,顺序扫描其后的100或1000个连续的文档。图\ref{fig:range_scan_performance}展示了评测结果。图中Couchstore(c)和ForestDB(c)代表已经完成压缩操作的Couchstore和ForestDB。在两个用例中,LevelDB都实现了最高的吞吐量。合并操作前Couchstore和ForestDB的范围扫描性能与随机读性能差不多,而压缩操作后的ForestDB的范围扫描性能几乎可以和LevelDB比肩。而压缩操作后的Couchstore对于范围扫描变现同样糟糕。这是因为Couchstore自身不维护块缓存,所以它交替访问文档和B$^+$树节点的磁盘物理空间是不相邻的。

\subsubsection{写缓存和HB$^+$字典树的影响}

我们下一步观察写缓存和HB$^+$字典树对ForestDB的独立影响。图\ref{fig:characteristics_of_forestdb_with_various_indexing_configurations}展示了在不同索引选项下ForestDB的总体吞吐和写放大情况:HB$^+$trie代表不使用写缓存的ForestDB,B$^+$tree代表不使用写缓存且使用原生B$^+$树作为索引结构的ForestDB,B$^+$tree+WB代表ForestDB使用原生B$^+$树作为索引结构且开启写缓存,HB$^+$trie+WB代表普通的ForestDB。

\begin{figure}[htbp]
    \centering
    \begin{overpic}[scale=0.6]{characteristics_of_forestdb_with_various_indexing_configurations.png}
        \put(15,-2){\scriptsize 写吞吐量}
        \put(45,-2){\scriptsize 读吞吐量}
        \put(86,-2){\scriptsize 写放大}
        \put(22,-6){\scriptsize (a)无序操作的读写吞吐量}
        \put(73,-6){\scriptsize (b)对B$^+$树归一化的写放大}
        \put(-3,20){\scriptsize \parbox[l]{1em}{每秒操作数}}
    \end{overpic}
    \\[3em]
	\caption{不同更新比率(a)和并发线程数(b)下的性能对比。\label{fig:characteristics_of_forestdb_with_various_indexing_configurations}}
\end{figure}

由于HB$^+$字典树可以降低每文档操作导致的磁盘访问量,使用HB$^+$字典树的读写整体吞吐量都比较高。图\ref{fig:characteristics_of_forestdb_with_various_indexing_configurations}b揭示了写缓冲通过缓冲索引节点更新有效的减少了写放大效应,对于B$^+$树或HB$^+$字典树都适用。Couchstore的写放大效应略优于B$^+$tree。这是因为相比于Couchstore,ForestDB在文档的元数据部分写入了更多的数据。

\subsubsection{局部性}

为了评估局部性负载下的性能变化,我们按照Zipf定律执行读取和更新操作。假设共有$N$项,则根据Zipf定律,第$p$项的频率函数为$f(p,s,N)=\frac{1}{p^s}/\sum_{n=1}^N\frac{1}{n^s}$,其中$s$为确定分布特征的参数。$p=1$的项具有最高的频率,随$p$值增大,频率逐渐降低。若$s$值趋近于$0$,频率分布则趋近于平均分布,而$s$越大,不同项之间频率差距越大。

我们随机选择10000个文档作为一个文档组,则2亿文档中有20000个文档组。然后我们使用$N=20000$的Zipf定律创建20000组频率,每组频率一一对应到随机选择的文档组。为了执行读取和更新操作,(1)基于预定频率随意选择一个文档组(2)在组中随机选择一个文档。

\begin{figure}[htbp]
    \centering
    \begin{overpic}[scale=0.6]{performance_comparison_according_to_the_value_of_s.png}
        \put(21,-3){\scriptsize 文档(\%)}
        \put(79,-3){\scriptsize $s$}
        \put(8,-6){\scriptsize (a)$N=20000$时Zipf累积分布函数图}
        \put(57,-6){\scriptsize (b)基于Zipf分布,含20\%更新操作的整体吞吐量}
        \put(-3,20){\scriptsize \parbox[l]{1em}{访问比(\%)}}
        \put(50,20){\scriptsize \parbox[l]{1em}{每秒操作数}}
    \end{overpic}
    \\[3em]
	\caption{不同$s$取值Zipf分布下的性能对比。\label{fig:performance_comparison_according_to_the_value_of_s}}
\end{figure}

图\ref{fig:performance_comparison_according_to_the_value_of_s}a是不同$s$取值下,每个文档访问累积分布函数图。我们可以通过不同的$s$取值控制负载局部性;负载局部性随$s$值的增加而提高。例如,当$s=1$时,大约80\%的访问集中在10\%的文档上。

不同类型局部性负载下总体性能对比结果展示在图\ref{fig:performance_comparison_according_to_the_value_of_s}b中。每种系统的吞吐两都随局部性的提高而提升,因为高局部性会提高操作系统页缓存的命中率,从而降低了访问操作造成的实际磁盘访问数。ForestDB随局部性增加的性能提升率远优于其他系统。这是因为ForestDB中每文档平均索引尺寸远小于其他系统,因此同容量的RAM可以缓存更多的索引数据。

\subsubsection{真实数据集结果}

我们最后使用由一个在线音乐流服务和垃圾网站集上获得的真实数据集评测所有的系统。表3描述了各个数据集的特征。每一个系统都使用数据集初始化,我们平均随机的执行读操作,并且同步执行更新操作。更新操作占操作总数的比为20\%,这与之前的评测相同。因为用户信息、播放列表、历史记录和状态数据集尺寸都适用于RAM,所以它们的整体吞吐量远高于Url数据集。这是因为整个DB文件都维护在操作系统页缓存中,因此读操作没有触发磁盘I/O。

\begin{figure}[htbp]
    \centering
    {
    \bfseries
    表3 \\
    真实数据集的特征 \\[1.5em]
    }
    \begin{tabular}{|p{4em}p{16em}p{4em}p{4em}p{4em}p{4em}|}
    \hline
    名称 & 描述 & 平均键长 & 平均文档尺寸 & 文档数量 & 数据集尺寸 \\
    \hline
    用户信息 & 使用用户ID索引的用户信息,例如用户名、电子邮箱和密码Hash值等。 & 33字节 & 230字节 & 4,839,099 & 1.7GB \\
    播放列表 & 使用ID索引的播放列表详细信息,例如拥有者ID、名称和修改时间等。 & 39字节 & 1,008字节 & 592,935 & 700MB \\
    历史记录 & 存储用户的活动记录,例如更新播放列表时记录时间戳和对应的活动ID。 & 29字节 & 2,772字节 & 1,694,980 & 4.7GB \\
    状态 & 使用用户电子邮箱或ID索引其当前的状态信息。 & 23字节 & 21字节 & 1,125,352 & 200MB \\
    Url & WEBSPAM-UK2007标准的Url值,垃圾网站集,Url作为键,文档随机产生。 & 111字节 & 1,024字节 & 105,896,555 & 112GB \\

    \hline
    \end{tabular}
\end{figure}

图\ref{the_overall_throughput_using_real_datasets}展现了测评结果。无论是何种数据集,Couchstore的整体性能几乎稳定。相反,LevelDB和RocksDB在文档总尺寸(键尺寸和文档内容尺寸之和)较小的时候表现出更好的性能。这是因为这些系统中每层的有序表同时存储键和文档内容,所以文档内容的尺寸影响着合并操作和压缩操作的开销。在Url数据集中,得益于RocksDB对I/O负载的优化,其性能优于LevelDB。

\begin{figure}[htbp]
    \centering
    \begin{overpic}[scale=0.6]{the_overall_throughput_using_real_datasets.png}
        \put(100,0){\scriptsize 100}
        \put(15,-1){\scriptsize 用户信息}
        \put(26,-1){\scriptsize 播放列表}
        \put(38,-1){\scriptsize 历史记录}
        \put(51,-1){\scriptsize 状态}
        \put(86,-1){\scriptsize Url}
        \put(-3,20){\scriptsize \parbox[l]{1em}{每秒操作数}}
        \put(67,20){\scriptsize \parbox[l]{1em}{每秒操作数}}
    \end{overpic}
    \\[1em]
	\caption{不同$s$取值Zipf分布下的性能对比。\label{fig:the_overall_throughput_using_real_datasets}}
\end{figure}

在所有数据集中,ForestDB的整体性能表现都要优于其他系统,因为HB$^+$字典树同事减少了从跟节点到叶的遍历路径长度和整体索引尺寸,索引的开销不仅由I/O设备负载,还通过内存分担。

\section{相关工作}

单节点键值存储可以分为两种类型:(1)位于内存的键值索引持久化存储,(2)常规键值存储引擎。前者键值存储引擎在持久化存储设备上维护键值对,但其全部或大部分索引数据位于内存。此类型的键值存储引擎通过降低单键内存尺寸,以达到将尽可能多的键值对数据在持久化存储设备中的位置缓存在内存中的目的。

Masstree是一个基于内存的键值存储引擎,它使用类字典树的多级联B$^+$树。为了最小化DRAM的取值时间(请求内存数据性能的关键因素),Masstree确定B$^+$树扇出时考虑了多核缓存行。更新被顺序记录在磁盘上以备份数据,而所有的索引节点都在内存中维护。虽然Masstree的上层设计类似HB$^+$字典树,但两者之间有很大区别。首先,Masstree主要的工作负载为所有的键值对都适合存储于RAM,因此它不需要考虑如何将索引节点组织和写入到磁盘,这是提高存储系统性能最棘手的问题。其次,Masstree没有提供任何关于平衡字典树倾斜的解决方案,它使用相对短,分布均匀的键规避这个问题。而无论是树遍历,还是空间开销,字典树倾斜问题已然成为了主要的障碍,需要解决它才能使之为通用的工作负载服务。

Lim et al.提出了SILT,这是一种部分位于内存的存储系统,它使用三层索引模式组织数据:日志存储层,Hash存储层和有序存储层。日志存储层,Hash存储层和有序存储层中每文档的平均内存占用逐次增加,而容量逐次减少。所有更新首先存储至日志存储层,随后刷新至Hash存储层。当Hash存储层中项数量超过某预定阈值时,这些项将被合并到基于字典树状结构的有序存储层。而有序存储层是不支持部分更新的,因此从Hash存储层到有序存储层的合并操作往往会引起全部项索引的调整。

FlashStore和SkimpyStash提出了将键值索引存储与闪存的方案。键值对被顺序的写入存储设备,位于内存的Hash表指向键值对在日志文件中的位置。两种方案的主要不同是内存用法。FlashStore为每一个键值日志维护单独的Hash表,而SkimpyStash中一部分键值日志以链表的形式存在于闪存中,每一个链表有一个专用的Hash表。这可以在每键值对内存占用量和磁盘访问量间进行置换。

上述的所有系统都在内存中维护它们的索引,因此从崩溃中恢复系统将引起大量的磁盘I/O,因为它们需要扫描所有的日志。此外,关闭或重启系统同样会触发大量的磁盘访问,因为所有位于内存的索引数据都应该回写到磁盘。这种开销难以被分布式NoSQL系统的通用单节点存储引擎接受。

另一种类型的键值存储引擎是通用后台引擎,它通常嵌入到其他需要键值索引功能的系统或应用中。通常它们的整体性能要弱于先前提到的那类键值存储,因为每一个更新操作的键值数据和相应的索引数据都存储在持久化设备中。然而它们可以从多种崩溃或失败情况中回复,利用较小的RAM可以提供更为通用合理的服务。

BerkeleyDB是应用最广泛的键值存储之一,它提供核心的后台键值存储功能。它基于B$^+$树索引,但也可以替换为其他索引结构(例如Hash)。支持多种事务特性,例如ACID(原子性、一致性、隔离性、持久性)和WAL。据信,BerkeleyDB为多种应用提供高性能低开销的键值存储服务。

LevelDB是另一个通用的键值存储,其概念借鉴于Google的BigTable。LevelDB基于LSM树的变种,其每个LSM树组件被成为层。Riak是基于Amazon Dynamo的NoSQL数据库,它作为LevelDB的存储引擎之一嵌入其中。Facebook发布了LevelDB的改良版RocksDB。它使用了多种不同的优化提高了很多系统性能指标,例如磁盘利用、读放大、压缩开销等。

Sears和Ramakrishnan支持bLSM,它被用于PNUTS(Yahoo分布式键值存储系统)和Walnut(Yahoo下一代弹性云存储系统)的后台存储。bLSM通过布隆过滤器提高LSM树的整体读取性能,同时采用新的名为Spring和Gear的压缩调度器以降低写延迟从而避免影响整体吞吐量。
\section{总结和未来展望}

本论文提出了一个为变长字符串键值对设计的单节点键值存储引擎——ForestDB。ForestDB使用HB$^+$字典树作为通用索引结构,其综合了Patricia字典树和B$^+$树。相比于树状索引结构,HB$^+$字典树具有低磁盘访问量和小空间占用量的优点。然而由于字典树基本上不是平衡的结构,在少数特殊情况下,该索引结构可能发生倾斜。为了解决这个问题,我们提出了解决方案。此外,ForestDB使用了结构化日志写入缓冲,以进一步降低文档更新产生的磁盘写入量。我们观察到,ForestDB的吞吐量显著高于其他键值存储方案。

因为目前ForestDB已经运行在传统文件系统上,而在文件系统上和ForestDB上检索的重复开销和元数据更新是无法避免的。为了避免这个问题,我们计划提出一个卷管理层,它使得ForestDB绕过文件系统层直接访问块设备。我们相信通过执行原始块I/O操作,整体性能会有大幅度提升。

\section{致谢}

本研究由韩国国家研究基金会(NRF)支持,有韩国政府提供资金(MSIP)(No. 2013R1A2A1A01016441)。本研究部分由MSIP/IITP[10041313, UX-oriented Mobile SW Platform]的ICT R\&D项目支持。
\section{参考文献}
\begin{flushleft}
[1] Couchbase NoSQL Database. [Online]. Available: http://www.couchbase.com/, 2011.

[2] J. Dean and S. Ghemawat. (2011). LevelDB: A fast and lightweight key/value database library by Google. [Online]. Available: https://github.com/google/leveldb

[3] RocksDB: a persistent key-value store for fast storage environments. [Online]. Available: http://rocksdb.org/, 2013.

[4] J. Han, E. Haihong, G. Le, and J. Du, "Survey on nosql database," in Proc. IEEE 6th Int. Conf. Pervasive Computing and Applications (ICPSCA), 2011, pp. 363–366.

[5] D. Comer,"Ubiquitous B-tree," ACM Comput. Surveys, vol. 11, no. 2, pp. 121–137, 1979.

[6] P. O’Neil, E. Cheng, D. Gawlick, andE. O’Neil, "The log-structured merge-tree (LSM-tree)," Acta Informatica, vol. 33, no. 4, pp. 351–385, 1996.

[7] M. A. Olson, K. Bostic, and M. I. Seltzer, "Berkeley DB," in Proc. USENIX Annu. Tech. Conf., FREENIX Track, 1999, pp. 183–191.

[8] P. Fruhwirt, M. Huber, M. Mulazzani, and E. R. Weippl, "InnoDB database forensics," in Proc. 24th IEEE Int. Conf. Adv. Inf. Netw. Appl., 2010, pp. 1028–1036.

[9] MongoDB. [Online]. Available: http://www.mongodb.com, 2009.

[10] SQLite4. [Online]. Available: https://sqlite.org/src4/doc/trunk/www/index.wiki, 2013.

[11] A. Lakshman and P. Malik, "Cassandra: A decentralized structured storage system," ACM SIGOPS Oper. Syst. Rev., vol. 44, no. 2, pp. 35–40, 2010.

[12] F. Chang, J. Dean, S. Ghemawat, W. C. Hsieh, D. A. Wallach, M. Burrows, T. Chandra, A. Fikes, and R. E. Gruber, "Bigtable: A distributed storage system for structured data," ACM Trans. Comput. Syst., vol. 26, no. 2, p. 4, 2008.

[13] R. Bayer and K. Unterauer, "Prefix B-trees," ACM Trans. Database Syst., vol. 2, no. 1, pp. 11–26, 1977.

[14] A. A. Moffat, T. C. Bell, and I. H. Witten, Managing Gigabytes: Compressing and Indexing Documents and Images. San Mateo, CA, USA: Morgan Kaufmann, 1999.

[15] D. R. Morrison, "Patricia: Practical algorithm to retrieve information coded in alphanumeric," J. ACM, vol. 15, no. 4, pp. 514–534, 1968.

[16] P. A. Bernstein and N. Goodman, "Concurrency control in distributed database systems," ACM Comput. Surveys, vol. 13, no. 2, pp. 185–221, 1981.

[17] J. C. Anderson, J. Lehnardt, and N. Slater, CouchDB: The Definitive Guide. O’Reilly Media, Inc., Sebastopol, CA, 2010.

[18] Apache CouchDB. [Online]. Available: http://couchdb.apache.org/, 2005.

[19] J.-S. Ahn. (2011). Third annual SIGMOD programming contest. [Online]. Available: http://goo.gl/yKUoiY

[20] Snappy: A fast compressor/decompressor. [Online]. Available: http://code.google.com/p/snappy/, 2011.

[21] G. K. Zipf, Human Behavior and the Principle of Least Effort, Cambridge, MA, Addison-Wesley Press, 1949.

[22] "WEBSPAM-UK2007. [Online]. Available: http://chato.cl/webspam/datasets/uk2007/.

[23] Y. Mao, E. Kohler, and R. T. Morris, "Cache craftiness for fast multicore key-value storage," in Proc. 7th ACM Eur. Conf. Comput. Syst., 2012, pp. 183–196.

[24] H. Lim, B. Fan, D. G. Andersen, and M. Kaminsky, "SILT: A memory-efficient, high-performance key-value store," in Proc. 23rd ACM Symp. Oper. Syst. Principles., 2011, pp. 1–13.

[25] B. Debnath, S. Sengupta, and J. Li, "FlashStore: High throughput persistent key-value store," in Proc. VLDB Endowment, vol. 3, no. 1-2, pp. 1414–1425, 2010.

[26] B. Debnath, S. Sengupta, and J. Li, "SkimpyStash: RAM space skimpy key-value store on flash-based storage," in Proc. ACM SIGMOD Int. Conf. Manage. Data., 2011, pp. 25–36.

[27] J. Gray, "The transaction concept: Virtues and limitations," in Proc. 7th Int. Conf. Very Large Data Bases, 1981, vol. 81, pp. 144–154.

[28] Riak. [Online]. Available: http://basho.com/riak/, 2009.

[29] G. DeCandia, D. Hastorun, M. Jampani, G. Kakulapati, A. Lakshman, A. Pilchin, S. Sivasubramanian, P. Vosshall, and W. Vogels, "Dynamo: Amazon’s highly available key-value store," ACM SIGOPS Oper. Syst. Rev., vol. 41, no. 6, pp. 205–220, 2007.

[30] R. Sears and R. Ramakrishnan, "bLSM: A general purpose log structured merge tree," in Proc. ACM SIGMOD Int. Conf. Manage. Data., 2012, pp. 217–228.

[31] B. F. Cooper, R. Ramakrishnan, U. Srivastava, A. Silberstein, P. Bohannon, H.-A. Jacobsen, N. Puz, D. Weaver, and R. Yerneni, "PNUTS: Yahoo!’s hosted data serving platform," in Proc. VLDB Endowment, 2008, vol. 1, no. 2, pp. 1277–1288.

[32] J. Chen, C. Douglas, M. Mutsuzaki, P. Quaid, R. Ramakrishnan, S. Rao, and R. Sears, "Walnut: A unified cloud object store," in Proc. ACM SIGMOD Int. Conf. Manage. Data., 2012, pp. 743–754.

\end{flushleft}

\end{document}
