\section{介绍}

近年来,数据管理和索引的需求愈发强烈。在诸如Facebook和Twitter这样的社交网络中,并发用户量和数据处理量越来越大,而数据本身也逐步趋向于非结构化以灵活的被应用分析。不幸的是,受限于可扩展性与严格的数据模型要求,关系型数据库难以解决这些问题。因此许多公司使用NoSQL作为关系型数据库的替代品。

虽然现有许多关于高级分片、数据复制和分布式缓存的NoSQL技术,但他们底层的持久化存储模块差别不大。单节点NoSQL通常使用键值存储来实现索引或获取无模式数据,其键与值通常是不定长的字符串。\footnote{在某些键值存储中,值可能是半结构化的数据(例如JSON文档),但它仍然可被看做字符串。}因为键值存储直接作用于磁盘(HHD)或固态硬盘(SSD)这样的通用块设备,其吞吐量和延迟决定了系统的整体性能。

键值存储的吞吐量和延迟受限于存储设备访问时间,其主要受两个因素影响:每次键值操作要访问的块数量及块访问模式。前者主要与索引结构的特点与逻辑设计有关,后者取决于含键值对在内的索引数据如何真正的写入或读取存储设备。

有两种广泛用于单机键值存储的索引结构:B$^+$树和结构化日志合并树(LSM树)。其中B$^+$树是最普及的索引结构之一,得益于它减少I/O操作的能力,它广泛应用于传统数据库。现代键值数据库(例如BerkeleyDB、Couchbase、InnoDB和MongoDB)使用B$^+$树作为底层存储。相反,LSM树由一组分层B$^+$树(或类似B$^+$树结构)组成,相较于传统B$^+$树,它通过牺牲读性能来提高写性能。很多近代系统(例如LevelDB、RocksDB、SQLite4、Cassandra和BigTable)使用LSM树或者其变种构建键值索引。

虽然目前为止,这些树型结构取得了成功,但是当它们使用键是变长字符串而非定长原始类型时,它们的性能将下降。如果节点尺寸是固定的,随着键尺寸的增大,节点扇出(例如节点中键指针数)将减少,为了保持同样的容量,树的高度将增大。另一方面,如果为了维持扇出数而增加节点尺寸,访问节点需要读写的块数量将会同比增加。不幸的是,树结构的高度和节点尺寸直接影响平均磁盘访问量和空间占用量,因此随着键长度的增加,索引的整体性能将下降。

为了解决这个问题,BerkeleyDB和LevelDB使用一种类似于前缀B$^+$树和预编码的前缀压缩技术。然而,这种方案大程度受限于键的模式。如果键在键空间内随机分布,键中剩余的未压缩部分仍然过长,因此前缀压缩技术的收益十分有限。为了索引变长字符串键,我们需要设计一种更有效的方法。

与此同时,在键值存储系统的设计中,块访问模式是另一个重要的因素。无论是HDD还是SSD,在操作块数相同的情况下,块读写顺序的分布很大程度的影响I/O性能。基于就地更新的策略的确能达到一个优秀的读性能,但是往往随之而来的是糟糕的写入延迟,对于近代写密集型场景来说,这是不可接受的。因此大多数数据库使用只追加或先行写入日志(WAL)的方式顺序的写入块。

这样的设计可以达到高写入吞吐量,但要付出合并和压缩(例如,垃圾收集)的开销。此类开销与平均每索引操作块访问数和索引结构空间占用量密切相关,因为在压缩过程中会执行许多合并操作。于是当长的键引起索引结构扇出降低时,开销将变大。

本文介绍了一个用于下一代Couchbase Server的单节点键值存储系统——ForestDB。为了在键尺寸较大和键随机分布的情况下均能在时间和空间角度高效的索引变长字符串键,我们提出了一种新颖的基于磁盘的索引结构——分层B$^+$字典树(HB$^+$字典树),作为ForestDB中的主键。HB$^+$字典树的逻辑结构基本上是Patricia字典树的一个变种,但它使用B$^+$树降低持久化存储设备上的块访问量。为了使读写操作都达到一个较高的吞吐量,并且支持高并发访问,更新索引操作使用只追加的方式写入存储设备。这也简化了磁盘结构以实现多版本并发控制(MVCC)。

我们以单机键值存储库的方式实现了ForestDB。我们使用真实数据集进行评测,结果显示ForestDB的平均每秒操作数显著高于LevelDB、RocksDB、Couchstore和Couchbase Server现有的键值存储模块。本文的其余部分安排如下。第二章概述了B$^+$树、前缀B$^+$树、LSM树和当前Couchstore的内部结构。第三章介绍了ForestDB的总体设计。第四章介绍评测结果,第五章介绍了相关工作。第六章总结全文。