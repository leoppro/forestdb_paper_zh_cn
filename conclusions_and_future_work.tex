\section{总结和未来展望}

本论文提出了一个为变长字符串键值对设计的单节点键值存储引擎——ForestDB。ForestDB使用HB$^+$字典树作为通用索引结构,其综合了Patricia字典树和B$^+$树。相比于树状索引结构,HB$^+$字典树具有低磁盘访问量和小空间占用量的优点。然而由于字典树基本上不是平衡的结构,在少数特殊情况下,该索引结构可能发生倾斜。为了解决这个问题,我们提出了解决方案。此外,ForestDB使用了结构化日志写入缓冲,以进一步降低文档更新产生的磁盘写入量。我们观察到,ForestDB的吞吐量显著高于其他键值存储方案。

因为目前ForestDB已经运行在传统文件系统上,而在文件系统上和ForestDB上检索的重复开销和元数据更新是无法避免的。为了避免这个问题,我们计划提出一个卷管理层,它使得ForestDB绕过文件系统层直接访问块设备。我们相信通过执行原始块I/O操作,整体性能会有大幅度提升。
